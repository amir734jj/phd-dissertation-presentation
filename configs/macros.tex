
\usetheme{Ilmenau}
% \usetheme{default}


\captionsetup[figure]{font=small}
\graphicspath{ {figures/} }

 %arrows is deprecated
\usetikzlibrary{positioning,calc,chains,positioning,arrows.meta,decorations.markings,datavisualization,matrix,shapes}

% Manual import hence all the files at the root level
\usetikzlibrary{ext.paths.ortho}
\usetikzlibrary{ext.node-families}

\newtheorem{plain}{Remark}


\addtobeamertemplate{navigation symbols}{}{%
    \usebeamerfont{footline}%
    \usebeamercolor[fg]{footline}%
    \hspace{1em}%
    \insertframenumber/\inserttotalframenumber
}

\setbeamercolor{footline}{fg=blue}
\setbeamerfont{footline}{series=\bfseries}

\newcommand{\enquote}[1]{``#1''}

\newcommand{\emptyline}{\vspace{0.3cm}}
% A subtitle is optional and this may be deleted
% \subtitle{Lattice based key-exchange}


\institute[UW-Milwaukee] % (optional, but mostly needed)
{
  Department of Computer Science\\
  \theuniversity
}
% - Use the \inst command only if there are several affiliations.
% - Keep it simple, no one is interested in your street address.

\date{Winter 2021}

\newcommand{\newlinevspace}{\vspace{\baselineskip}}



% - Either use conference name or its abbreviation.
% - Not really informative to the audience, more for people (including
%   yourself) who are reading the slides online

\subject{Theoretical Computer Science}
% This is only inserted into the PDF information catalog. Can be left
% out. 

% If you have a file called "university-logo-filename.xxx", where xxx
% is a graphic format that can be processed by latex or pdflatex,
% resp., then you can add a logo as follows:

\pgfdeclareimage[height=0.5cm]{university-logo}{university-logo-uwm}
\logo{\pgfuseimage{university-logo}}

\algdef{SE}[DOWHILE]{Do}{DoWhile}{\algorithmicdo}[1]{\algorithmicwhile\ #1}%

\newcommand{\AlgMultiline}[1]{
\parbox[t]{\dimexpr\linewidth-\csname ALG@tlm\endcsname+\algorithmicindent}{\raggedright\hangindent\algorithmicindent%
            #1 }}
            
\setbeamertemplate{mini frames}{}

\AtBeginSection[]
{
    \begin{frame}
        \frametitle{Table of Contents}
        \tableofcontents[currentsection]
    \end{frame}
}


\newcommand{\customorder}{{quasi total order}}
\newcommand{\Customorder}{{Quasi total order}}
\newcommand{\CustomorderSymbol}{{\prec}}