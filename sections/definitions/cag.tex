\subsection*{Circular AG}{}

\begin{frame}{Extensions Overview}{Classical AG extensions}
     \begin{description}
        \item $\checkmark$ Classical attribute grammar
        \item $\checkmark$  Remote attribute grammars
        \item $\square$  \alert{Circular attribute grammars}
        \item $\square$  Circular remote attribute grammars
    \end{description}
\end{frame}

\note[itemize]{
\item Next, we are going to describe circular attribute grammars
}

\begin{frame}{Circular Attribute Grammar}{Classical AG + allowing circularity}
\begin{definition}
An attribute grammar is \alert{circular} if there exists a derivation tree of the context-free grammar whose \alert{attribute dependency has a cycle}.
\end{definition}
\end{frame}

\note[itemize]{
\item Traditionally, even in the Knuth paper circularities in attribute grammar have been treated as an error and have been avoided
\item But in mathematics or even programming its natural to define things circularly or recursively 
\item Detecting if attribute grammar is circular is not straightforward. It's actually NP-complete.
\item Basically, we need to test if there is a cycle in dependency graph in any of possible derivations, but a context-free grammar may have an infinite number of derivations
}


\begin{frame}{Circular AG Evaluation}{Ascending chain condition}

Farrow showed that \alert{in certain situations} AG with circular dependencies can be evaluated using fixed point semantics, and this is called \enquote{circular attribute grammar extension}


\newlinevspace

$\Rightarrow$ Require that the domain of all attributes involved in cyclic chains can be arranged in a \alert{lattice of finite height} and that all uses in involved in the cycle to be \alert{monotone}

\[ x_{i+1} = f(x_i) \text{ where } x_0 = \bot \]

\end{frame}

\note[itemize]{
\item How to ensure that evaluation of circular attribute grammar terminates?
\item Farrow described a class of circular attribute grammars where it uses fixed-point semantics to ensure that attribute grammar can be evaluated in finite number of iterations
\item $x_0$ is a bottom value of the lattice and the height of the lattice is finite so if we evaluate the cycle over and over again, we will eventually get a fixed-point value
}



\begin{frame}{\Customorder{}}{Precursor to define schedule for circular attribute grammar}
    Formally, a \customorder{} relation $\lesssim$ satisfies the following properties:

\begin{enumerate}
    \item For all $x,y$ and $z$, if $x\lesssim y$ and $y\lesssim z$ then $x\lesssim z$ (\alert{transitivity})
    \item For all $x, y$, then $x\lesssim y$ or $y\lesssim x$ must be true or both (strong connectedness or \alert{total})
\end{enumerate}
\end{frame}

\note[itemize]{
\item The precursor to define schedule for circular attribute grammars is to define some kind of order, this order cannot be \enquote{total-order} because we may have cycles
\item Therefore, we defined \customorder{} as an order such that it is transitive and total
\item We use less-sim symbol $\lesssim$ for \customorder{}
}

\begin{frame}{Total order of partition}{\Customorder{} as total order on the partition}

The following set of binary \customorder{} relationship: $S=\{ a \lesssim b$, $b \lesssim a$,  $b \lesssim c$, $c \lesssim c \}$ can be expressed as total order on the partition of $S$: $\big \{ \{ a, b\} < \{ c \} \big \}$.

\begin{figure}
    \centering
\begin{tikzpicture}[scale=0.70]
\begin{scope}[every node/.style={circle,thick,draw}]
    \node (A) at (0,0) {a};
    \node (B) at (3,0) {b};
    \node (C) at (6,0) {c};
\end{scope}

\begin{scope}[>={Stealth[black]},
              every node/.style={fill=white,circle},
              every edge/.style={draw=black,very thick}]
    \path [->] (A) edge[bend right=60] (B);
    \path [->] (B) edge[bend right=60] (A);
    \path [->] (B) edge[bend right=0] (C);
    \path [->] (A) edge[bend right=-90, dotted] (C);
    \path [->] (C) edge[loop below, in=-50,out=-130, looseness=8] (C);
\end{scope}
\end{tikzpicture}
\end{figure}

\end{frame}

\note[itemize]{
\item \Customorder{} can be defined as a total order on the partition of a set
\item This means that \customorder{} on the instantiated rules is isomorphic to the \enquote{total-order} on a partition of instantiated rules
\item By partition, we mean an unordered set. So there is no order between instantiated rules that belong to the same inner set
\item Also notice in this picture we have loops, these are elements that are in a cycle
}

\begin{frame}{Schedule For CAG (This Thesis)}{Schedule for circular attribute grammar}

\begin{definition}\label{def:cag-schedule-definition}
A \emph{schedule} for a circular attribute grammar is a \customorder{} ($\lesssim$) on the set of instantiated rules.
\end{definition}

\newlinevspace

Schedule is valid when ($a < b$ is $a \lesssim b \wedge b \not \lesssim a$):

% TODO
\[\begin{cases}
      \hat{r_i} < \hat{r_j},    & \text{whenever we have a simple dependency}  \\
      \hat{r_i} \lesssim \hat{r_j}, & \text{whenever we have a monotone dependency}  
    \end{cases}\]
    
\end{frame}


\note[itemize]{
\item Previously we defined schedule for classical attribute grammar using $\mathit{DO}$ and $\mathit{UO}$
\item However in circular attribute grammar we have to ensure simple or non-monotone dependencies should not be involved in a cycle
\item The key is distinguishing between simple (or non-monotone) and monotone uses 
\item Schedule for circular attribute grammar is a \customorder{} on the instantiated rules
\item Schedule is valid when dependencies are respected, meaning that simple or non-monotone uses are not involved in a cycle
\item Also, note that less than symbol $<$ here does not mean total order, instead it means something is less-sim $\lesssim$ another but not the other way around
\item I also did not use the word: \enquote{strict}
}


% trivial schedule can go to where you talk about circular stuff
\begin{frame}{Trivial Schedule}{Observation}

Only when \alert{all uses are monotone}

$$ \Big\{ \{ \hat{r_0}, \dots, \hat{r_7} \} \Big \}$$

This happens when for all $\hat{r_1}, \hat{r_2} \in \hat{R}$, $\hat{r_1} \lesssim \hat{r_2} \wedge \hat{r_2} \lesssim \hat{r_1}$.

\end{frame}

\note[itemize]{
\item In circular and circular remote attribute grammars, we can have a trivial schedule
\item Trivial schedule is when we have all monotone functions
\item Basically all instantiated rules may belong to the same level
\item So we just evaluate all instantiated rules over and over again until a fixed point is reached
}

\begin{frame}{$l$-Ordered Circular Attribute Grammars}{Definition of $l$-ordered for CAG}

$l$-ordered Circular Attribute Grammars requires:
\begin{itemize}
    \item Summary graph to be \customorder{}
    \item Augmented dependency for each production is a \customorder{} such that:
    \begin{itemize}
        \item All dependencies involved in cycles must be monotone
        \item Restricted \customorder{} of the augmented dependency graph for a non-terminal must be the same as the \customorder{} for the summary graph for that non-terminal
    \end{itemize}
\end{itemize}


\end{frame}

\note[itemize]{
\item In $l$-ordered circular attribute grammar, we are required to have a summary graph that is a \customorder{}
\item For every augmented dependency graph we can form a \customorder{} and all dependencies in the cycle have to be monotone 
\item Also, if we restrict the \customorder{} of the augmented dependency graph to just one of the non-terminals, it should be the same \customorder{} for the summary graph
}

\begin{frame}[fragile=singleslide]{CAG Example}{CAG example with 2 independent cycles}



\begin{multicols}{2}
\begin{Verbatim}[fontsize=\small]
S -> A A
    A0.i1 = A1.s1
    A0.i2 = A1.s2
 
    A1.i1 = A0.s1
    A1.i2 = A0.s2

    S.s = A0.s1 + A0.s2

A -> a
    A.s1 = A.i1
    A.s2 = A.i2





\end{Verbatim}
\end{multicols}
    
\end{frame}

\note[itemize]{
\item As an exercise, let's analyze this circular attribute grammar
\item Notice that there are two independent cycles involving $i_1$, $s_1$ and another cycle involving $i_2$, $s_2$
\item To resolve $A_1.s_1$ we need to resolve $A_1.i_1$ and to resolve that we need to resolve $A_0.s_1$ and to resolve that we need to resolve $A_1.s_1$, hence the cycle as we write to $A_1.s_1$ using $A_1.s_1$.
\item There is a similar cycle involving attributes $i_2$ and $s_2$
}


\begin{frame}{CAG $l$-ordered}{ Summary graph for CAG example}
    Summary graph that is \customorder{} for non-terminal $A$

    \[ \bigg\{  \big \{i_1, s_1\big\} < \big\{ i_2, s_2 \big\} \bigg \} \]


	\begin{figure}[htbp]
		\centering
		\begin{tikzpicture}[->,>=Stealth,auto,scale=0.6,shorten >= 4pt,
			thick,main node/.style={draw, rectangle, align=center}]

			\node[main node,text width=1.5cm] (1) at (0,0)    {$i_1$};
			\node[main node,text width=1.5cm, anchor=north west] (2) at(1.north east) {$i_2$};
			\node[main node,text width=1.5cm, anchor=south west] (3) at(2.north east) {$s_1$};
			\node[main node,text width=1.5cm, anchor=north west] (4) at(3.north east) {$s_2$};
		
			\draw[blue, dash dot] (1.south) to[out=-90, in=-120,looseness=1] (3.south);
			\draw[red, dash dot] (2.south) to[out=-60, in=-120,looseness=1] (4.south);

   			\draw[black, dash dot] (3.north) to[out=120, in=60,looseness=1.2] (2.north);

      		\draw[blue, dash dot] (3.north) to[out=60, in=60,looseness=1.2] (1.north);
                \draw[red, dash dot] (4.north) to[out=90, in=120,looseness=1.5] (2.north);
		\end{tikzpicture} 
		
		\caption{$\mathit{SDG}_A$}
	\end{figure}

\end{frame}
\note[itemize]{
\item Let's propose a summary graph for non-terminal $A$ that captures all possible dependencies among its attribute
\item We can see we have two independent cycles, if we did not have the connection (colored in black) it would not be \customorder{} (or total)
\item This is the \customorder{} of the summary graph as a total order on the partition
}



\begin{frame}{CAG $l$-ordered}{ Augmented dependency graph for CAG example}

    $\mathit{ADG}_{S \to AA}$: Augmented dependency graph for $S \to A A$

	\begin{figure}[htbp]
		\centering
		\begin{tikzpicture}[->,>=Stealth,auto,scale=0.35,shorten >= 4pt,
			thick,main node/.style={draw, rectangle, align=center}]
			
			\node[main node,text width=1cm] (1) at (5,14)    {$S.s$};
			
			
			\node[main node,text width=1cm] (2) at (-8,0)    {$A_0.i_1$};
			\node[main node,text width=1cm, anchor=north west] (3) at(2.north east) {$A_0.i_2$};
			\node[main node,text width=1cm, anchor=south west] (4) at(3.north east) {$A_0.s_1$};
			\node[main node,text width=1cm, anchor=north west] (5) at(4.north east) {$A_0.s_2$};
			
			\node[main node,text width=1cm] (6) at (8,0)    {$A_1.i_1$};
			\node[main node,text width=1cm, anchor=north west] (7) at(6.north east) {$A_1.i_2$};
			\node[main node,text width=1cm, anchor=south west] (8) at(7.north east) {$A_1.s_1$};
			\node[main node,text width=1cm, anchor=north west] (9) at(8.north east) {$A_1.s_2$};
			
			\draw[black] (4.north) to[out=90, in=-120,looseness=1] (1.south);
			\draw[black] (5.north) to[out=90, in=-60,looseness=1] (1.south);

			\draw[blue] (8.north) to[out=60, in=90,looseness=1.2] (2.north);
			\draw[red] (9.north) to[out=90, in=120,looseness=1] (3.north);

			\draw[blue] (4.north) to[out=60, in=120,looseness=1] (6.north);
			\draw[red] (5.north) to[out=60, in=120,looseness=1] (7.north);
   
			\draw[blue, dash dot] (2.south) to[out=-90, in=-120,looseness=1] (4.south);
			\draw[red, dash dot] (3) to[out=-90, in=-120,looseness=1] (5.south);

			\draw[blue, dash dot] (6.south) to[out=-90, in=-120,looseness=1] (8.south);
			\draw[red, dash dot] (7.south) to[out=-90, in=-120,looseness=1] (9.south);
   
			\draw[black, dash dot] (4.north) to[out=90, in=60,looseness=1.7] (3.north);
			\draw[black, dash dot] (8.north) to[out=90, in=60,looseness=1.7] (7.north);
   
		\end{tikzpicture} 
		
	\end{figure}
	

    
\end{frame}
\note[itemize]{
\item Notice this dependency graph is definitely a \customorder{}
\item There are two independent cycles: red and blue
\item These cycles became total thanks to the two edges (colored in black) from $A_0.s_1$ to $A_0.i_2$ and similarly $A_1.s_1$ to $A_1.i_2$, these two edges came from the summary dependency graph of non-terminal $A$
\item If we restrict the augmented dependency graph to just non-terminal $A$, we will get back the same \customorder{} that was given in the summary graph of $A$, hence this circular attribute grammar is $l$-ordered
}

\begin{frame}{\Customorder{} in $\mathit{ADG}$}

    \Customorder{} in the $\mathit{ADG}_{S \to AA}$
    
\[ \bigg \{  \big \{ A_0.i_1, A_0.s_1, A_1.i_1, A_1.s_1 \big \} < \big \{ A_0.i_2, A_0.s_2, A_1.i_2, A_1.s_2 \big \} < S.s \bigg \} \]


\end{frame}

\note[itemize]{
\item This is the \customorder{} of the augmented dependency graph we saw in the previous slide as a total order on the partition of the set
\item Notice that we have two independent cycles and they appear in-order 
}

\begin{frame}{$l$-Ordered Circular Attribute Grammars}{Extending definition of $l$-ordered to circular AGs (CAG)}
    
    \begin{lemma}\label{cag-lordered-wellformed}
$l$-ordered circular attribute grammars are well-formed.
\end{lemma}
\newlinevspace
    Constructive proof with contradictions in case analysis.
    
\end{frame}

\note[itemize]{
\item Lastly, we \textbf{proved} in the thesis that the definition of $l$-ordered can be extended to circular attribute grammars and its well-formed
}