
\subsection*{Circular Remote AG}{}

\begin{frame}{Extensions Overview}{Classical AG extensions}
     \begin{description}
        \item $\checkmark$ Classical attribute grammar
        \item $\checkmark$  Remote attribute grammars
        \item $\checkmark$ Circular attribute grammars
        \item $\square$   \alert{Circular remote attribute grammars}
    \end{description}
\end{frame}

\note[itemize]{
\item Next, we are going to describe circular remote attribute grammars
}


\begin{frame}{Circular Remote Attribute Grammar}{Definition: Circular + Remote AG}

\begin{definition}
A circular remote attribute grammar is an \alert{extension of remote attribute grammars} and has the same form as remote attribute grammars, except \alert{certain attributes are circular} and \alert{uses of attribute occurrences in functions are declared monotone in some arguments}.
\end{definition}
\end{frame}

\note[itemize]{
\item This is the informal definition of circular remote attribute grammar or CRAG
\item It is a merge of two previous two extensions: circular and remote into one
\item This is the most accommodating extension to describe circularly defined problems over a tree
}

% TODO
\begin{frame}{Circular Remote Attribute Grammar}{Applications}
Applications for CRAG:

\begin{itemize}
    \item First and Follow sets
    \item Unification (in type inference)
	\item Sub-typing with circular types (a class extending its own generic parameter)
	\item Detecting circular class extension in COOL semantic analyzer (e.g. A extends B, B extends C, C extends A)
 \item Program control-flow analysis
\end{itemize}
\end{frame}

\note[itemize]{
\item There are many applications for a circular remote attribute grammar including First and Follow in parsing, unification in type inference, detecting circularly defined class hierarchy during type checking, and control flow analysis
% \item COOL is a programming language (classroom object-oriented language)
}


\begin{frame}[fragile=singleslide]{Circular Remote Attribute Grammar}{Evaluation strategy}
Hedin in \cite{10.1016/j.scico.2005.06.005} used fixed-point loops with \alert{demand evaluation} to evaluate circular remote attribute grammar.

\begin{Verbatim}[fontsize=\small]
    do {
    
        // Evaluate rules (again?)
        
    } while (currentValue \supset prevValue)
\end{Verbatim}

\newlinevspace

But, what about \alert{schedule}? how to formalize the validity of the schedule for CRAG?
\end{frame}

\note[itemize]{
\item In CRAG, similar to circular attribute grammars we need a fixed point loop for evaluation
\item This is the basic setup of a fixed-point loop introduced by Hedin
\item Basically, we hold on to the previous value and then run the evaluation and then check whether the new value is a superset, NOT superset-equal to the previous value. And if that is the case then we re-run the loop.
}



\begin{frame}{Circular Remote Attribute Grammar}{Evaluation Terminating? Monotonicity}

\begin{block}{Remark}
The partial field write rule as defined in the definition of remote attribute grammar is a \alert{monotone} use.
\end{block}

\end{frame}

\note[itemize]{
\item Adding a monotonicity constraint ensures that fixed-point evaluation terminates
\item However, partial field writes are monotone, this is because object fields are \enquote{set} so they can be involved in a cycle
\item The only thing remaining is ensuring that arguments of semantic functions are monotone in a classical rule
}






\begin{frame}{Circular Remote Attribute Grammar}{Schedule Definition}
    
\begin{definition}
A schedule for a circular remote attribute grammar is a \alert{\customorder{}} $\lesssim$ on the \alert{instantiated rules}.
\end{definition}


Schedule is valid when ($a < b$ is $a \lesssim b \wedge b \not \lesssim a$):

% TODO: the same as 38
\[\begin{cases}
      \hat{r_i} < \hat{r_j},    & \text{whenever a simple dependency} \\
      \hat{r_i} \lesssim \hat{r_j}, & \text{whenever a monotone dependency}
    \end{cases}\]

    And for any two rules $\hat{r}_i = (v \texttt{=} w.f)$, $\hat{r}_j = (w'.f \sqsupseteq v' ) \in R(t)$ if these rules are \alert{potentially} referring to the same object then $\hat{r}_j \lesssim \hat{r}_i$ 

\end{frame}

\note[itemize]{
\item Schedule for CRAG is a \customorder{} of instantiated rules
\item Schedule is valid when dependencies defined using $\mathit{DO}$ and $\mathit{UO}$ are respected meaning that the definition of attribute instance precedes the use of the same attribute instance
\item And similarly writes to field of an object should precede the read of the field of the same object
}



% keep this :)
\begin{frame}[fragile=singleslide]{Example}{CRAG Example}

\begin{multicols}{3}
\begin{Verbatim}[fontsize=\small]
S -> A B
    local l
    l = A.r
    B.i = l
    A.i = B.s
    S.x = l.f
A -> a
    object o
    o.f <- A.i
    A.r = o


B -> b
    local l
    l = B.i
    B.s = l.f
\end{Verbatim}
\end{multicols}

\newlinevspace

$\Rightarrow$ Notice the \alert{cycle} involving reading and writing of the same object

\end{frame}

\note[itemize]{
\item This is an example of circular remote attribute grammar
\item Notice $o.f$ gets $A.i$ using partial write and $A.i$ itself gets $B.s$ using classical rule
\item Then $B.s$ gets $l.f$ where $l$ is a local attribute assigned
\item Basically, we write to $o.f$ using the value of $o.f$. Hence a cycle.
}


% kill this
\begin{frame}[fragile=singleslide]{Example}{Instantiated CRAG with trivial derivation}

\begin{multicols}{3}
\begin{Verbatim}[fontsize=\small]
n0: S -> A B
    local l0
    r0: l0 = n1.r
    r1: n2.i = l0
    r2: n1.i = n2.s
    r3: n0.x = l0.f
n1: A -> a
    object o0
    r4: o0.f <- n1.i
    r5: n1.r = o0


n2: B -> b
    local l1
    r6: l1 = n2.i
    r7: n2.s = l1.f
\end{Verbatim}
\end{multicols}

Schedule: 

\[
     \Big \{ \{ \hat{r}_5 \} < \{ \hat{r}_0 \} < \{ \hat{r}_1 \} < \{ \hat{r}_6 \} < \{ \hat{r}_4 , \hat{r}_7, \hat{r}_2 \} < \{ \hat{r}_3 \}  \Big \}
\]

\end{frame}

\note[itemize]{
\item This is the instantiated form of the circular remote attribute grammar we introduced in the previous slide
\item The corresponding schedule is at the bottom
\item Remember that \customorder{} is a total-order on a partition, hence why we used less-than $<$ here
}




% kill this
\begin{frame}[fragile=singleslide]{Example}{CRAG Example with non-monotone function $h$}

\begin{multicols}{3}
\begin{Verbatim}[fontsize=\small]
n0: S -> A B
    local l
    r0: l = h(n1.r)
    r1: n2.i = h(l)
    r2: n1.i = n2.s
    r3: n0.x = h(l.f)
n1: A -> a
    object o
    r4: o.f <- n1.i
    r5: n1.r = h(o0)


n2: B -> b
    local l
    r6: l = h(n2.i)
    r7: n2.s = l.f
\end{Verbatim}
\end{multicols}

\newlinevspace

The \textbf{trivial schedule} \alert{does not work} for this example because of $h$

\end{frame}

\note[itemize]{
\item If we introduce non-monotone function $h$ in a cycle, this means that the trivial schedule is no longer valid
}


\begin{frame}{$l$-Ordered Circular Remote Attribute Grammars}{Extending definition of $l$-ordered to circular remote AGs (CRAG)}
    
    \begin{lemma}\label{crag-lordered-wellformed}
$l$-ordered circular remote attribute grammars are well-formed.
\end{lemma}
\newlinevspace
    Proof by applying \enquote{fiber construction} and \enquote{fiber approximation} to proof of $l$-ordered circular attribute grammars
    
\end{frame}

\note[itemize]{
\item Lastly, we showed in the thesis that definition of $l$-ordered can be extended to circular remote attribute grammars thanks to \enquote{fiber-construction} and \enquote{fiber-approximation}
}