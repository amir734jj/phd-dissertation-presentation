
\section{Definitions}

\subsection*{CFG}{}

\begin{frame}{Context-Free Grammar (CFG)}{Foundation that attribute grammars built upon}

Context-free grammar consists of:
\begin{itemize}
    \item set of \alert{non-terminals}
    \begin{itemize}
        \item one of which is designated as the start variable
    \end{itemize}
    \item set of \alert{terminals}
    \item set of \alert{productions}
\end{itemize}
\end{frame}

\note[itemize]{
\item What is grammar or more specifically context-free grammar?
\item It is a tuple of a set of non-terminals, a set of terminals, and a set of productions.
}

\begin{frame}[fragile=singleslide]{Example}{Grammar for a simple language}

\begin{multicols}{2}
\begin{Verbatim}[fontsize=\scriptsize]
program -> block

block -> "begin" decls stmts "end"

decls -> decls decl
    |   /* empty */

decl -> ID ":" type ";"

type -> "Integer"
    |   "String"

stmts -> stmts stmt
    |   /* empty */

stmt -> expr ":=" expr
    |   block ";"

expr -> INT_CONSTANT
    |   STRING_CONSTANT
    |   ID
\end{Verbatim}
\end{multicols}


Source: \cite{Boyland1998AnalyzingDN}

\end{frame}

\note[itemize]{
\item This is an example of grammar that defines a simple program that declares some variables and assigns them to an expression
\item Notice that it starts from a program and goes to a block
\item Block consists of declarations and assignments
}


\begin{frame}[fragile=singleslide]{CFG Example}{A derivation of this simple language}

    \begin{columns}
    \column{0.5\textwidth}
\begin{Verbatim}[fontsize=\scriptsize,numbers=left,xleftmargin=5mm]
begin
   x : String;
   y : Integer;
	
   x := z;
   y := "hello world!";
end
\end{Verbatim}

    \column{0.5\textwidth}
\alert{Semantic Errors}
\begin{itemize}
    \item Use of undeclared variable?
    \item Type mismatch?
    \item Unused variable?
\end{itemize}
    \end{columns}

\newlinevspace
\newlinevspace

$\Rightarrow$ How to \alert{identity} these issues \alert{using} \emph{attribute grammars} ...?
\end{frame}

\note[itemize]{
\item This is a program OR a derivation of the context-free grammar that we saw in the previous slide
\item Notice there are multiple semantic errors in this program, can you identify them?
\item Attribute grammars can help solving tree problems like these
}




\begin{frame}{Classical Attribute Grammar}{CFG + Rules}


A classical attribute grammar is a \alert{CFG grammar} with the following added features:

\begin{itemize}
    \item Each non-terminal $X$ has a set of attributes $A(X)$
    \item $A(X)$ has two disjoint subsets
    \begin{itemize}
        \item $S(X)$, synthesized attributes, which are passed up the tree
        \item $I(X)$, inherited attributes which are passed down the tree
    \end{itemize}
    \item Set of local attributes associated with each production
    \item Each production of the grammar has a set of \alert{semantic rules}
\end{itemize}

\end{frame}

\note[itemize]{
\item What is a classical attribute grammar?
\item It is a tuple of:
\begin{itemize}
    \item Context-free grammar
    \item Set of attributes for each non-terminal
    \item Set of local attributes
    \item Each production has a set of semantic rules that defines attributes for the LHS non-terminal of the production

\end{itemize}
}





\begin{frame}[fragile=singleslide]{Semantic Function}{Identity function syntactic sugar}

\begin{verbatim}
A -> B c       // CFG production
    A.s = B.s  // <-- semantic function: A.s = id(B.s)
               // with one argument
               // returns the argument
\end{verbatim}

This particular semantic function is called an \alert{identity function}

\end{frame}

\note[itemize]{
\item This is just a syntax sugar we will be using throughout this presentation
\item \textit{id} here means an identity function
\item Also, in a production, we may say a \textit{parent node} and by that we mean LHS non-terminal, we may also say \textit{child nodes} or children and by that we mean RHS non-terminals
}









% \begin{frame}[fragile=singleslide]{Example}{Classical attribute grammar to find semantic errors in a program}


% \begin{multicols}{2}
% \begin{Verbatim}[fontsize=\fontsize{1}{1}\selectfont]
% program -> block
% 	block.env = empty_env()
% 	program.msgs = block.msgs
	
% block -> "begin" decls stmts "end"
% 	decls.envin = block.env
% 	stmts.env = decls.envout
% 	decls.uin = stmts.used
% 	block.used = decls.uout
% 	block.msgs = decls.msgs ++ stmts.msgs
	
% decls ->
% 	decls.envout = decls.envin
% 	decls.uout = decls.uin
% 	decls.msgs = { }

% decls -> decls decl
%   decls1.envin = decls0.envin
%   decl.envin = decls1.envout
%   decls0.envout = decl.envout
%   decl.uin = decls0.uin
%   decls1.uin = decl.uout
%   decls0.uout = decls1.uout
%   decls0.msgs = decls1.msgs ++ decl.msgs

% decl -> id ":" type ";"
%   decl.envout = add_env(<id, type.shape>, decl.envin)
%   decl.uout = decl.uin - id
%   decl.msgs = if id in decl.uin then
%                 { }
%               else
%                 { "unused: " + id }

% type -> "Integer"
%   type.shape = INT_SHAPE

% type -> "String"
%   type.shape = STR_SHAPE

% stmt ->
%   stmts.used = { }
%   stmts.msgs = { }

% stmts -> stmts stmt
%   stmts1.env = stmts0.env
%   stmt.env = stmts0.env
%   stmts0.used = stmts1.used ++ stmt.used
%   stmts0.msgs = stmts1.msgs ++ stmt.msgs

% stmt -> block ";"
%   block.env = stmt.env
%   stmt.used = block.used
%   stmt.msgs = block.msgs

% stmts -> expr ":=" expr ";"
%   expr1.env = stmt.env
%   expr2.env = stmt.env
%   stmt.used = expr1.used ++ expr2.used
%   stmt.msgs = (if expr1.shape != expr2.shape then
%                 { "type mismatch" }
%               else
%                 { }) ++ expr.msgs

% expr -> INT_CONSTANT
%   expr.shape = INT_SHAPE
%   expr.used = { }
%   expr.msgs = { }

% expr -> STR_CONSTANT
%   expr.shape = STR_SHAPE
%   expr.used = { }
%   expr.msgs = { }

% expr -> id
%   local shape = lookup(id, expr.env)
%   expr.shape = shape
%   expr.used = { id }
%   expr.msgs = if shape == NOT_FOUND then
%                 { id + " not declared" }
%               else
%                 { }
% \end{Verbatim}
% \end{multicols}

% \end{frame}

% \note[itemize]{
% \item This is a classical attribute grammar that detects: type mismatch, unused variables, and use of undeclared variable that we identified in the previous slide
% \item $\mathit{msgs}$ is a synthesized collection attribute of error messages
% \item $\mathit{env}$ is an inherited attribute that denotes the set of variables declared and available to us
% \item $\mathit{used}$ is a set of all variables used as an expression
% }

% maybe delete
\begin{frame}{Instantiated Attribute Grammar}{Attribute Grammar + Derivation = Instantiated AG}
Given a \alert{derivation} of CFG grammar, an attribute grammar becomes \alert{instantiated}:
\begin{itemize}
    \item Set of \alert{instantiated semantic rules} where each attribute \alert{occurrence} is replaced with an \alert{attribute instance}
\end{itemize}

\end{frame}

\note[itemize]{
\item Combining an attribute grammar and a derivation results in an instantiated attribute grammar where instead of attribute occurrences we now have attribute instances.
\item Similarly, instead of semantic rules we have instantiated semantic rules.
}

\subsection*{Evaluations}

\begin{frame}{Evaluation}{Core concept}

\begin{itemize}
    \item Evaluation is a process of finding the \alert{values} of \alert{attribute instances}.

    \item Two types of evaluation:
\begin{itemize}
    \item \alert{Demand} evaluation (slow, memory inefficient)
    \item \alert{Static} evaluation (faster, less memory footprint)
\end{itemize}
\end{itemize}

\end{frame}

\note[itemize]{
\item We observed that value or semantic information is stored in attributes
\item The \enquote{values} of attributes are the result of the evaluation of rules associated with productions of the grammar
\item And the evaluation is a process of finding the values of attribute instances
\item There are two types of evaluations: static and demand
\item Demand is a kind of evaluation where the evaluator demands each attribute instance value (usually root node's synthesized attributes), and then the evaluator will find the dependent attribute instances needed to evaluate those, and so on. 
\item In demand evaluation we don't know if the evaluation is going to succeed or not until the evaluation finishes and it requires a runtime dependency analysis to find the dependencies
\item Static evaluation is the opposite. There is no need to keep track of runtime dependency.
}


\begin{frame}{Demand Evaluation}{Straightforward evaluation method}
\begin{definition}
Demand evaluation is a kind of evaluation where each attribute instance access requires a call to evaluate the corresponding instantiated semantic rule that defines it.
\end{definition}

\begin{alertblock}{Observation}
In demand evaluation, we do not know whether the evaluation will succeed or not until the evaluation finishes
\end{alertblock}

% fix this
% \begin{alertblock}{Observation}
% Demand evaluation 
% \end{alertblock}

\end{frame}

\note[itemize]{
\item Demand evaluation replaces access to the value of an attribute instance with a function call that defines the attribute in the first place
\item Basically, during the evaluation runtime, the evaluator tries to figure out which rule it should evaluate next
\item This form of evaluation is simple to understand and implement but it has some downsides, more specifically not knowing whether the evaluation will terminate or not and a need to keep track of runtime dependencies during the evaluation runtime which can be costly both in terms of space and time complexity
}

%  move to where we talk about demand previously
\begin{frame}{Demand Evaluation}{Pros and cons}
Pros:
\begin{itemize}
    \item Can benefit form \alert{caching to prevent re-evaluation}
    \item Easy to implement
\end{itemize}

Cons:
\begin{itemize}
    \item Does not detect cycles before the evaluation begins and \alert{may not terminate}
    \item Space complexity if \alert{caching} is used: $\mathcal{O}(|\hat{V}|)$
    \item Time complexity of: $\mathcal{O}(| \hat{V} | \times | \hat{R} |)$ (for classical AG)
\end{itemize}
\end{frame}

\note[itemize]{
\item In demand evaluation we have to use a cache to avoid exponential time complexity because everything will get evaluated over and over again and caching will impact the space complexity
\item Also, it does not detect cycles in the dependency graph before the evaluation begins so we do not know if evaluation is going to terminate or not. This is not ideal.
}


% maybe get rid of
\begin{frame}[fragile=singleslide]{Example}{Example of classical attribute grammar}
    
\begin{multicols}{3}
\begin{verbatim}
S -> A
  A.i1  = S.in
  A.i2  = A.s1
  S.out = A.s2




A -> A A
  A1.i1 = A0.i1
  A1.i2 = A1.s1
  A0.s1 = A1.s2
  A2.i1 = A0.i2
  A2.i2 = A2.s1
  A0.s2 = A2.s2

A -> 'a'
  A.s1 = A.i1 + 1
  A.s2 = A.i2 + 2
    
    
    
    
\end{verbatim}
\end{multicols}

Source: \cite{10.1145/225540.225544}
    
\end{frame}

\note[itemize]{
\item This is an example of classical attribute grammar
\item Notice root non-terminal $S$ has two attributes, one inherited and one synthesized
\item Similarly, non-terminal $A$ has two inherited and two synthesized attributes
}

% get rid of possibly
\begin{frame}[fragile=singleslide]{Derivation}{4 nodes in the tree}

\[
\lefteqn{\underbrace{\phantom{S \rightarrow A}}_{n_0}} S \rightarrow
\lefteqn{\overbrace{\phantom{A \rightarrow A A}}^{n_1}} A \rightarrow 
\lefteqn{\underbrace{\phantom{A A \rightarrow \texttt{a}}}_{n_2}} A A \rightarrow \texttt{a}
\lefteqn{\overbrace{A \rightarrow \texttt{aa}}^{n_3}}
\]

\begin{center}
\scalebox{0.75}{\begin{forest}
  [
    $n_0$, name=n0
    [ $n_1$
        [$n_2$ [$\texttt{a}$]]
        [$n_3$ [$\texttt{a}$]]
    ]
  ]
\end{forest}}
\end{center}

\end{frame}

\note[itemize]{
\item This is a derivation of attribute grammar we saw in the previous slide
\item This derivation results in 4 nodes in the syntax tree
}

\begin{frame}[fragile=singleslide]{Example}{Instantiated attribute grammar}


\begin{multicols}{3}
\begin{Verbatim}[fontsize=\scriptsize]
n0: S -> A
  r0 : n1.i1  = n0.in
  r1 : n1.i2  = n1.s1
  r2 : n0.out = n1.s2




n1: A -> A A
  r3 : n2.i1 = n1.i1
  r4 : n2.i2 = n2.s1
  r5 : n1.s1 = n2.s2
  r6 : n3.i1 = n1.i2
  r7 : n3.i2 = n3.s1
  r8 : n1.s2 = n3.s2

n2: A -> 'a'
  r9 : n2.s1 = n2.i1 + 1
  r10: n2.s2 = n2.i2 + 2
    
n3: A -> 'a'
  r11: n3.s1 = n3.i1 + 1
  r12: n3.s2 = n3.i2 + 2
\end{Verbatim}
\end{multicols}

The sequence of steps needed to evaluate $\alert{n_0.\mathit{out}}$:
$\alert{n_0.\mathit{out}} \to \alert{r_2}{:} n_o.\mathit{out} = n_1.s_2 \to \alert{r_8}{:} n_1.s_2 = n_3.s_2 \to \alert{r_{12}}{:}n_3.s_2 = \dots$

\end{frame}

\note[itemize]{
\item This is the instantiated form of an attribute grammar using that derivation we saw in the previous slide
\item $n_0$, $n_1$, $n_2$ and $n_3$ are tree nodes
\item $n_i$ followed by a dot followed by attribute name is called an attribute instance
\item Also, notice the sequence of steps a demand evaluator needs to take during the evaluation runtime to evaluate the root node's synthesized attribute $n_0.\mathit{out}$
\item This sequence of instantiated rules used to evaluate all attribute instances is called a schedule
\item We have to run a topological sort on an instantiated attribute grammar to find the schedule
}

\begin{frame}{Schedule}{Total order of instantiated rules}
What makes a schedule (total order on instantiated rules) valid?

\newlinevspace

$\Rightarrow$ \alert{No use of attribute instance before it is defined first}
\newlinevspace

Schedule can be formally defined using $\mathit{DO}$ and $\mathit{UO}$ notation.

\begin{itemize}
    \item \alert{$\mathit{DO}(r)$}: defining occurrences
    \item \alert{$\mathit{UO}(r)$}: used occurrences
\end{itemize}

\[r{:}\; X.a = Y.b\;\;\;\; \mathit{DO}(r) = \{X.a\}\;\mathit{UO}(r) = \{Y.b\}\]

\end{frame}

\note[itemize]{
\item In a valid schedule, we have to arrange the instantiated rules such that no attribute instance is used before it is defined first.
\item So definition of an attribute instance has to precede the use.
\item Schedule can be defined formally using $\mathit{DO}$ and $\mathit{UO}$ sets which are called defining occurrences and used occurrences respectively.
\item Basically if you are using an attribute instance, the rule that defines it must be scheduled earlier that the rule that uses it
}


\begin{frame}{Schedule for Classical Attribute Grammar}{Schedule and what makes it valid}
    \begin{itemize}
        \item Schedule for classical attribute grammar is a \alert{total order on instantiated rules}

        \item Schedule is \alert{valid} when ($a < b$ is a total order):

    \begin{itemize}
        \item $\hat{r_i} < \hat{r_j} \text{  whenever } \hat{v}_k \in \mathit{DO}(\hat{r_i}) \wedge \hat{v}_k \in \mathit{UO}(\hat{r_j})$
    \end{itemize}
    \end{itemize}

\end{frame}


\note[itemize]{
\item Schedule for classical attribute grammar is a total order of instantiated rules
\item Schedule is valid when dependencies defined using $\mathit{DO}$ and $\mathit{UO}$ are respected meaning that the definition of attribute instance strictly precedes the use of the same attribute instance
\item We use $\hat{r}$ \enquote{hat} to represent instantiated semantic rules
}

\begin{frame}{Schedule Evaluation}{Finding the order of evaluation of instantiated rules before evaluation runtime}

Given this \emph{schedule}, evaluate the AG:

\begin{equation}
\begin{split}
\mathit{schedule} = \Big \{\hat{r}_0 < \hat{r}_3 < \hat{r}_9 < \hat{r}_4 < \hat{r}_{10} < \hat{r}_5 < \hat{r}_1 < \hat{r}_6 \\
< \hat{r}_{11} < \hat{r}_7 < \hat{r}_{12} < \hat{r}_8 < \hat{r}_2 \Big \}    
\end{split}
\end{equation}

% obvious way but not required, fix this
\begin{alertblock}{Observation}
Finding the schedule requires topological sort of attribute instances: $\mathcal{O}(| V +  E|)$ where $V = \hat{V}$ and $E = |\hat{V}  \times  \hat{R}|$
\end{alertblock}
\end{frame}

\note[itemize]{
\item Schedule evaluation is a type of dynamic evaluation where we find the schedule beforehand and then during the evaluation runtime, we evaluate the instantiated rules according to the schedule.
\item This is the schedule for the instantiated attribute grammar we saw in the previous slide
}



% kill this slide
\begin{frame}{Schedule Evaluation}{Pros and cons}
Pros:
\begin{itemize}
    \item If scheduler finds a schedule, then evaluation \alert{will terminate}
    \item Easy to implement
    \item \alert{No possibility of re-evaluation}
    \item Better space complexity
    \item Time complexity of: $\mathcal{O}(| \hat{R} |)$ (for classical AG)
\end{itemize}

Cons:
\begin{itemize}
    \item Schedule \alert{works only for one derivation} of attribute grammar
\end{itemize}
\end{frame}

\note[itemize]{
\item Schedule evaluator runs in a linear time for classical attribute grammar
\item It has a better space complexity as no runtime dependency information is needed
\item But schedule only works for one derivation, and if we are given a different derivation then we need to find a different schedule. So this is not ideal.
}






\begin{frame}{Static Evaluation}{Alternative evaluation method: Constructed independent of a particular derivation}

\begin{itemize}
    \item What if there was a way to \alert{statically} evaluate attribute grammars?
    \begin{itemize}
        \item Generate evaluator \alert{once} and would \alert{work for all possible derivations} of an attribute grammar.
    \end{itemize}
    \item What \alert{constraints} need to be true for such an attribute grammar to have a static evaluator?
    \item Can we guarantee the evaluation always \alert{terminate}?
    \item How to verify if static evaluation is \alert{valid}?
\end{itemize}
\end{frame}

\note[itemize]{
\item What if there was a better way to evaluate attributes? some kind of static evaluator that would create the schedule in the runtime without runtime keeping track of dependencies and this static evaluator would work for all possible derivations of this context-free grammar? this would be ideal
}




\begin{frame}{$l$-Ordered Attribute Grammars}{Key to static evaluation}

Kastens introduced \enquote{ordered attribute grammars}:

\newlinevspace 

Basically, for each symbol of the grammar, a total-order over the associated attributes can be defined, such that given any derivation of this CFG, the attributes can be evaluated in that order.

\end{frame}
\note[itemize]{
\item First, lets talk about $l$-ordered evaluation class of attribute grammars
\item Kastens introduced ordered attribute grammars and this evaluation class of attribute grammars says that one can construct an algorithm or static schedule to evaluate the attributes given any derivation of an ordered attribute grammar.
\item In $l$-ordered evaluation class, attributes can always be evaluated in that particular order
\item There are non-circular AGs where there is no fixed order, but interesting non-circular AGs can be evaluated this way
}

\begin{frame}{$l$-ORD}{How to check AG is $l$-ordered}

How to check attribute grammar is $l$-ordered:

\begin{itemize}
    \item for each non-terminal in the grammar
    \item \enquote{guess} a summary graph or order of attributes for a non-terminal
    \item construct an augmented dependency graph
    \item ensure check the augmented dependency graph is non-circular
\end{itemize}

\newlinevspace

This is a NP problem, more specifically \alert{NP-complete} \cite{ENGELFRIET1982283}

\end{frame}

\note[itemize]{
\item The problem of determining a non-circular AG is $l$-ordered is actually NP-complete because we have to guess an order of attributes for each non-terminal and then construct augmented dependency graph and check if its non-circular
}



% determinining AG is l-ordered is NP complete, but OAG is a subset
\begin{frame}{Ordered AG Test}{Test if visit sequence evaluator exists (efficiently)}
$l$-ordered membership test is \alert{NP-complete} \cite{ENGELFRIET1982283} so it's common/practical to use an \alert{OAG subset test} test which is a greedy algorithm that runs in \alert{polynomial time}.

\end{frame}


\note[itemize]{
\item Testing for $l$-ordered is NP-complete and not practical at all
\item OAG test finds a subset of $l$-ordered and its membership test runs in a polynomial time
}



\subsection*{Remote AG}{}

\begin{frame}{Extensions Overview}{Classical AG extensions}
     \begin{description}
        \item $\checkmark$ Classical attribute grammar
        \item $\square$  \alert{Remote attribute grammars}
        \item $\square$  Circular attribute grammars
        \item $\square$  Circular remote attribute grammars
    \end{description}
\end{frame}

\note[itemize]{
\item Next, we are going to describe remote attribute grammars
}


% no definitions
% high-level, no G S I L
% 
\begin{frame}{Remote Attribute Grammar (RAG)}{Classical AG + allowing references}

% todo
\begin{itemize}
    \item Introduced by introduced separately in both \cite{Boyland_1995, Boyland05remoteattribute},  and \cite{4236994589494506a212280893694207}
    \item Extension of classical attribute grammar by adding a set of \alert{objects} and \alert{fields}
    \item In addition to the capabilities of classical AG, it allows the passing of object references and object fields as attributes.
    \item In RAG, we can access attribute values that are defined \alert{non-locally} whereas in classical AG all attributes are defined and used \alert{locally}
\end{itemize}


\end{frame}

\note[itemize]{
\item Remote attribute grammars were introduced separately by Prof. Boyland and Prof. Hedin. Basically, it extends classical attribute grammar and introduces a set of objects and a set of fields
\item This means that we now have objects and we can pass them around by reference
\item Furthermore, it gives the user more flexibility as we are now allowed to read and write non-local attributes as opposed to only local read and local writes in classical attribute grammar
}






\begin{frame}{Schedule for RAG Observation}{Partial field write}

It is valid to have \alert{multiple partial field write} for the same field of an object. This is \alert{unlike classical rule} where we can define an attribute instance \alert{once}

\[ w'.f \sqsupseteq v' \]

\newlinevspace

This makes it challenging as the schedule for RAG should order instantiated rules such that \alert{read of the object's field} should be done when its value is \alert{final} (i.e. after all writes are done)

\end{frame}

\note[itemize]{
\item Remote attribute grammar extension introduces two new forms of semantic rule in addition to the classical form
\item Partial write of object field, and read of the object field
\item Note that attribute instance has to be defined once in a classical form and then we can read it
\item But the object field may be written to multiple times, so we need to make sure we arrange the rules in the schedule in such a way that the reading of the object's field happens when the value is final
\item Objects are assumed instantiated before the evaluation begins
\item We use $\sqsupseteq$ square-super-set-eq symbol to indicate that $\hat{v}$ is contributing to the object field value of $\hat{w}.f$
}





\begin{frame}{RAG Semantic and $l$-ordered in RAG}{Potentially infinite attribute instances}
    
\begin{itemize}
    \item Express \alert{semantics of remote attribute grammars} in classical attribute grammar using \enquote{fiber construction}
    \item This can lead to \alert{infinite} attribute instances (\alert{improper attribute grammar})
    \begin{itemize}
        \item Objects can be attributes and attributes can be on objects too
    \end{itemize}
    \item Boyland \cite{Boyland05remoteattribute} introduced a process called \enquote{fiber approximation} that results in the representation of the RAG in classical AG with a \alert{finite number of attribute instances}
    \item This process can be used to define $l$-ordered for remote attribute grammars by applying it to the definition of $l$-ordered for classical attribute grammars 
\end{itemize}

\end{frame}

\note[itemize]{
\item Semantic of remote attribute grammar can be defined in terms of a classical attribute grammar, this process is called fiber construction
\item However, the problem is objects can be attributes and attributes can be objects so we end up with an improper attribute grammar with an infinite number of attribute instances
\item Professor Boyland defined fiber approximation which can do just that in a finite number of attribute instances
\item Subsequently, we can apply that process to the definition of $l$-ordered for classical attribute grammar to define ordered remote attribute grammar
}

\begin{frame}{Schedule for RAG}{Schedule and what makes schedule valid}

        \begin{itemize}
        \item Schedule for remote attribute grammar is a \alert{total order on instantiated rules}

        \item Schedule is \alert{valid} when ($a < b$ is a total order):

    \begin{itemize}
        \item $\hat{r_i} < \hat{r_j} \text{  whenever } \hat{v}_k \in \mathit{DO}(\hat{r_i}) \wedge \hat{v}_k \in \mathit{UO}(\hat{r_j})$
        \item And for any two rules $\hat{r}_i = (v \texttt{=} w.f)$, $\hat{r}_j = (w'.f \sqsupseteq v' ) \in R(t)$ if these rules are \alert{potentially} referring to the same object then $\hat{r}_j < \hat{r}_i$ 
    \end{itemize}
    \end{itemize}

\end{frame}

\note[itemize]{
\item Schedule for remote attribute grammar is a total order of instantiated rules
\item Schedule is valid when dependencies defined using $\mathit{DO}$ and $\mathit{UO}$ are respected meaning that the definition of attribute instance strictly precedes the use of the same attribute instance
\item Similarly, writes to the field of an object should strictly precede the read of the field of the same object
}

% kill this
% \begin{frame}[fragile=singleslide]{Example}{Remote attribute grammar to find semantic errors in a program}

% \begin{centering}
% \begin{multicols}{2}
% \begin{Verbatim}[fontsize=\fontsize{5.5}{6}\selectfont]
% program -> block
%   block.scope = ROOT_SCOPE

% block -> "begin" decls stmts "end"
%   local scope = { decls: { }, enclosing: block.scope }
%   decls.scope = scope
%   stmts.scope = scope

% decls ->

% decls -> decls decl
%   decls1.scope = decls0.scope
%   decl.scope = decls0.scope

% decl -> id ":" type ";"
%   local d = { shape: type.shape, col: false } 
%   decl.scope.decls <- { <id, d> }
%   if not d.used then
%     msgs <- { id + " not used" }

% type -> "Integer"
%   type.shape = INT_SHAPE

% type -> "String"
%   type.shape = STR_SHAPE

% stmts -> 

% stmts -> stmts stmt
%   stmts1.scope = stmts0.scope
%   stmt.scope = stmts0.scope

% stmt -> block ";"
%   block.scope = stmt.scope

% stmt -> expr := expr
%   expr1.scope = stmt.scope
%   expr2.scope = stmt.scope
%   if expr1.shape != expr2.shape then
%     msgs <- { "type mismatch" }

% expr -> INT_CONSTANT
%   expr.shape = INT_SHAPE

% expr -> STRING_CONSTANT
%   expr.shape = STR_SHAPE

% expr -> id
%   local decl = lookup(id, expr.scope)
%   expr.shape = decl.shape
%   if decl == NOT_FOUND then
%     msgs <- { id + " not declared" }
%   else
%     decl.used <- true
% \end{Verbatim}
% \end{multicols}
% \end{centering}

% \end{frame}

% \note[itemize]{
% \item This is the program analysis example using remote attribute grammar
% \item Notice how much smaller it is now thanks to remote access and eliminating the need to pass attributes up and then down
% }





\subsection*{Circular AG}{}

\begin{frame}{Extensions Overview}{Classical AG extensions}
     \begin{description}
        \item $\checkmark$ Classical attribute grammar
        \item $\checkmark$  Remote attribute grammars
        \item $\square$  \alert{Circular attribute grammars}
        \item $\square$  Circular remote attribute grammars
    \end{description}
\end{frame}

\note[itemize]{
\item Next, we are going to describe circular attribute grammars
}

\begin{frame}{Circular Attribute Grammar}{Classical AG + allowing circularity}
\begin{definition}
An attribute grammar is \alert{circular} if there exists a derivation tree of the context-free grammar whose \alert{attribute dependency has a cycle}.
\end{definition}
\end{frame}

\note[itemize]{
\item Traditionally, even in the Knuth paper circularities in attribute grammar have been treated as an error and have been avoided
\item But in mathematics or even programming its natural to define things circularly or recursively 
\item Detecting if attribute grammar is circular is not straightforward. It's actually NP-complete.
\item Basically, we need to test if there is a cycle in dependency graph in any of possible derivations, but a context-free grammar may have an infinite number of derivations
}


\begin{frame}{Circular AG Evaluation}{Ascending chain condition}

Farrow showed that \alert{in certain situations} AG with circular dependencies can be evaluated using fixed point semantics, and this is called \enquote{circular attribute grammar extension}


\newlinevspace

$\Rightarrow$ Require that the domain of all attributes involved in cyclic chains can be arranged in a \alert{lattice of finite height} and that all uses in involved in the cycle to be \alert{monotone}

\[ x_{i+1} = f(x_i) \text{ where } x_0 = \bot \]

\end{frame}

\note[itemize]{
\item How to ensure that evaluation of circular attribute grammar terminates?
\item Farrow described a class of circular attribute grammars where it uses fixed-point semantics to ensure that attribute grammar can be evaluated in finite number of iterations
\item $x_0$ is a bottom value of the lattice and the height of the lattice is finite so if we evaluate the cycle over and over again, we will eventually get a fixed-point value
}


\begin{frame}{Simple and Monotone use}
    \begin{equation}
  \begin{gathered}
\mathit{SUO}(r) = \{ v_j \mid M_{g j} = \texttt{false} \text{ for }  1 \leq j \leq k  \text{ in classical rule } \\ v_0 \texttt{=} g(v_1, \dots, v_k) \}  \\ 
\mathit{\mathit{MUO}}(r) = \mathit{UO}(r) - \mathit{SUO}(r) \\
\mathit{UO}(r) = \mathit{SUO}(r) \cup \mathit{\mathit{MUO}}(r) 
  \end{gathered}
\end{equation}
\end{frame}

\note[itemize]{
\item Previously we defined schedule for classical attribute grammar using $\mathit{DO}$ and $\mathit{UO}$
\item However in circular attribute grammar we have to ensure simple or non-monotone dependencies should not be involved in a cycle
\item The key is distinguishing between simple (or non-monotone) and monotone uses by splitting $\mathit{UO}$ into $\mathit{SUO}$ and $\mathit{MUO}$ 
}


\begin{frame}{\Customorder{}}{Precursor to define schedule for circular attribute grammar}
    Formally, a \customorder{} relation $\lesssim$ satisfies the following properties:

\begin{enumerate}
    \item For all $x,y$ and $z$, if $x\lesssim y$ and $y\lesssim z$ then $x\lesssim z$ (\alert{transitivity})
    \item For all $x, y$, then $x\lesssim y$ or $y\lesssim x$ must be true or both (strong connectedness or \alert{total})
\end{enumerate}
\end{frame}

\note[itemize]{
\item The precursor to define schedule for circular attribute grammars is to define some kind of order, this order cannot be \enquote{total-order} because we may have cycles
\item Therefore, we defined \customorder{} as an order such that it is transitive and total
\item We use less-sim symbol $\lesssim$ for \customorder{}
}

\begin{frame}{Total order of partition}{\Customorder{} as total order on the partition}

The following set of binary \customorder{} relationship: $S=\{ a \lesssim b$, $b \lesssim a$,  $b \lesssim c$, $c \lesssim c \}$ can be expressed as total order on the partition of $S$: $\big \{ \{ a, b\} < \{ c \} \big \}$.

\begin{figure}
    \centering
\begin{tikzpicture}[scale=0.70]
\begin{scope}[every node/.style={circle,thick,draw}]
    \node (A) at (0,0) {a};
    \node (B) at (3,0) {b};
    \node (C) at (6,0) {c};
\end{scope}

\begin{scope}[>={Stealth[black]},
              every node/.style={fill=white,circle},
              every edge/.style={draw=black,very thick}]
    \path [->] (A) edge[bend right=60] (B);
    \path [->] (B) edge[bend right=60] (A);
    \path [->] (B) edge[bend right=0] (C);
    \path [->] (A) edge[bend right=-90, dotted] (C);
    \path [->] (C) edge[loop below, in=-50,out=-130, looseness=8] (C);
\end{scope}
\end{tikzpicture}
\end{figure}

\end{frame}

\note[itemize]{
\item \Customorder{} can be defined as a total order on the partition of a set
\item This means that \customorder{} on the instantiated rules is isomorphic to the \enquote{total-order} on a partition of instantiated rules
\item By partition, we mean an unordered set. So there is no order between instantiated rules that belong to the same inner set
\item Also notice in this picture we have loops, these are elements that are in a cycle
}

\begin{frame}{Schedule For CAG (This Thesis)}{Schedule for circular attribute grammar}

\begin{definition}\label{def:cag-schedule-definition}
A \emph{schedule} for a circular attribute grammar is a \customorder{} ($\lesssim$) on the set of instantiated rules.
\end{definition}

\newlinevspace

Schedule is valid when ($a < b$ is $a \lesssim b \wedge b \not \lesssim a$):

% TODO
\[\begin{cases}
      \hat{r_i} < \hat{r_j},    & \text{whenever } \hat{v}_k \in \mathit{DO}(\hat{r_i}) \wedge \hat{v}_k \in \mathit{SUO}(\hat{r_j}) \\
      \hat{r_i} \lesssim \hat{r_j}, & \text{whenever } \hat{v}_k \in \mathit{DO}(\hat{r_i}) \wedge \hat{v}_k \in \mathit{MUO}(\hat{r_j})  
    \end{cases}\]
    
\end{frame}


\note[itemize]{
\item Schedule for circular attribute grammar is a \customorder{} on the instantiated rules
\item Schedule is valid when dependencies are respected, meaning that simple or non-monotone uses are not involved in a cycle
\item Also, note that less than symbol $<$ here does not mean total order, instead it means something is less-sim $\lesssim$ another but not the other way around
\item I also did not use the word: \enquote{strict}
}


% trivial schedule can go to where you talk about circular stuff
\begin{frame}{Trivial Schedule}{Observation}

Only when \alert{all uses are monotone}

$$ \Big\{ \{ \hat{r_0}, \dots, \hat{r_7} \} \Big \}$$

This happens when for all $\hat{r_1}, \hat{r_2} \in \hat{R}$, $\hat{r_1} \lesssim \hat{r_2} \wedge \hat{r_2} \lesssim \hat{r_1}$.

\end{frame}

\note[itemize]{
\item In circular and circular remote attribute grammars, we can have a trivial schedule
\item Trivial schedule is when we have all monotone functions
\item Basically all instantiated rules may belong to the same level
\item So we just evaluate all instantiated rules over and over again until a fixed point is reached
}

\begin{frame}{$l$-Ordered Circular Attribute Grammars}{Definition of $l$-ordered for CAG}

$l$-ordered Circular Attribute Grammars requires:
\begin{itemize}
    \item Summary graph to be \customorder{}
    \item Augmented dependency for each production is a \customorder{} such that:
    \begin{itemize}
        \item All dependencies involved in cycles must be monotone
        \item Restricted \customorder{} of the augmented dependency graph for a non-terminal must be the same as the \customorder{} for the summary graph for that non-terminal
    \end{itemize}
\end{itemize}


\end{frame}

\note[itemize]{
\item In $l$-ordered circular attribute grammar, we are required to have a summary graph that is a \customorder{}
\item For every augmented dependency graph we can form a \customorder{} and all dependencies in the cycle have to be monotone 
\item Also, if we restrict the \customorder{} of the augmented dependency graph to just one of the non-terminals, it should be the same \customorder{} for the summary graph
}

\begin{frame}[fragile=singleslide]{CAG Example}{CAG example with 2 independent cycles}



\begin{multicols}{2}
\begin{Verbatim}[fontsize=\small]
S -> A A
    A0.i1 = A1.s1
    A0.i2 = A1.s2
 
    A1.i1 = A0.s1
    A1.i2 = A0.s2

    S.s = A0.s1 + A0.s2

A -> a
    A.s1 = A.i1
    A.s2 = A.i2





\end{Verbatim}
\end{multicols}
    
\end{frame}

\note[itemize]{
\item As an exercise, let's analyze this circular attribute grammar
\item Notice that there are two independent cycles involving $i_1$, $s_1$ and another cycle involving $i_2$, $s_2$
\item To resolve $A_1.s_1$ we need to resolve $A_1.i_1$ and to resolve that we need to resolve $A_0.s_1$ and to resolve that we need to resolve $A_1.s_1$, hence the cycle as we write to $A_1.s_1$ using $A_1.s_1$.
\item There is a similar cycle involving attributes $i_2$ and $s_2$
}


\begin{frame}{CAG $l$-ordered}{ Summary graph for CAG example}
    Summary graph that is \customorder{} for non-terminal $A$



	\begin{figure}[htbp]
		\centering
		\begin{tikzpicture}[->,>=Stealth,auto,scale=0.6,shorten >= 4pt,
			thick,main node/.style={draw, rectangle, align=center}]

			\node[main node,text width=1.5cm] (1) at (0,0)    {$i_1$};
			\node[main node,text width=1.5cm, anchor=north west] (2) at(1.north east) {$i_2$};
			\node[main node,text width=1.5cm, anchor=south west] (3) at(2.north east) {$s_1$};
			\node[main node,text width=1.5cm, anchor=north west] (4) at(3.north east) {$s_2$};
		
			\draw[black, dash dot] (1.south) to[out=-90, in=-120,looseness=1] (3.south);
			\draw[black, dash dot] (2.south) to[out=-90, in=-120,looseness=1] (4.south);

   			\draw[black, dash dot] (3.north) to[out=90, in=60,looseness=1] (2.north);
		\end{tikzpicture} 
		
		\caption{$\mathit{SDG}_A$}
	\end{figure}

\end{frame}
\note[itemize]{
\item Let's propose a summary graph for non-terminal $A$ that captures all possible dependencies among its attribute 
}



\begin{frame}{CAG $l$-ordered}{ Augmented dependency graph for CAG example}

    $\mathit{ADG}_{S \to AA}$: Augmented dependency graph for $S \to A A$

	\begin{figure}[htbp]
		\centering
		\begin{tikzpicture}[->,>=Stealth,auto,scale=0.35,shorten >= 4pt,
			thick,main node/.style={draw, rectangle, align=center}]
			
			\node[main node,text width=1cm] (1) at (5,14)    {$S.s$};
			
			
			\node[main node,text width=1cm] (2) at (-8,0)    {$A_0.i_1$};
			\node[main node,text width=1cm, anchor=north west] (3) at(2.north east) {$A_0.i_2$};
			\node[main node,text width=1cm, anchor=south west] (4) at(3.north east) {$A_0.s_1$};
			\node[main node,text width=1cm, anchor=north west] (5) at(4.north east) {$A_0.s_2$};
			
			\node[main node,text width=1cm] (6) at (8,0)    {$A_1.i_1$};
			\node[main node,text width=1cm, anchor=north west] (7) at(6.north east) {$A_1.i_2$};
			\node[main node,text width=1cm, anchor=south west] (8) at(7.north east) {$A_1.s_1$};
			\node[main node,text width=1cm, anchor=north west] (9) at(8.north east) {$A_1.s_2$};
			
			\draw[black] (4.north) to[out=90, in=-120,looseness=1] (1.south);
			\draw[black] (5.north) to[out=90, in=-60,looseness=1] (1.south);

			\draw[blue] (8.north) to[out=60, in=90,looseness=1.2] (2.north);
			\draw[red] (9.north) to[out=90, in=120,looseness=1] (3.north);

			\draw[blue] (4.north) to[out=60, in=120,looseness=1] (6.north);
			\draw[red] (5.north) to[out=60, in=120,looseness=1] (7.north);
   
			\draw[blue, dash dot] (2.south) to[out=-90, in=-120,looseness=1] (4.south);
			\draw[red, dash dot] (3) to[out=-90, in=-120,looseness=1] (5.south);

			\draw[blue, dash dot] (6.south) to[out=-90, in=-120,looseness=1] (8.south);
			\draw[red, dash dot] (7.south) to[out=-90, in=-120,looseness=1] (9.south);
   
			\draw[black, dash dot] (4.north) to[out=90, in=60,looseness=1.5] (3.north);
			\draw[black, dash dot] (8.north) to[out=90, in=60,looseness=1.5] (7.north);
   
		\end{tikzpicture} 
		
	\end{figure}
	

    
\end{frame}
\note[itemize]{
\item Notice this dependency graph is definitely a \customorder{}
\item There are two independent cycles: red and blue
\item These cycles became total thanks to the two edges (colored in black) from $A_0.s_1$ to $A_0.i_2$ and similarly $A_1.s_1$ to $A_1.i_2$, these two edges came from the summary dependency graph of non-terminal $A$
\item If we restrict the augmented dependency graph to just non-terminal $A$, we will get back the same \customorder{} that was given in the summary graph of $A$, hence this circular attribute grammar is $l$-ordered
}

\begin{frame}{\Customorder{} in $\mathit{ADG}$}

    \Customorder{} in the $\mathit{ADG}_{S \to AA}$
    
\[ \bigg \{  \big \{ A_0.i_1, A_0.s_1, A_1.i_1, A_1.s_1 \big \} < \big \{ A_0.i_2, A_0.s_2, A_1.i_2, A_1.s_2 \big \} < S.s \bigg \} \]


\end{frame}

\note[itemize]{
\item This is the \customorder{} of the augmented dependency graph we saw in the previous slide as a total order on the partition of the set
\item Notice that we have two independent cycles and they appear in-order 
}

\begin{frame}{$l$-Ordered Circular Attribute Grammars}{Extending definition of $l$-ordered to circular AGs (CAG)}
    
    \begin{lemma}\label{cag-lordered-wellformed}
$l$-ordered circular attribute grammars are well-formed.
\end{lemma}
\newlinevspace
    Constructive proof with contradictions in case analysis.
    
\end{frame}

\note[itemize]{
\item Lastly, we \textbf{proved} in the thesis that the definition of $l$-ordered can be extended to circular attribute grammars and its well-formed
}

\subsection*{Circular Remote AG}{}

\begin{frame}{Extensions Overview}{Classical AG extensions}
     \begin{description}
        \item $\checkmark$ Classical attribute grammar
        \item $\checkmark$  Remote attribute grammars
        \item $\checkmark$ Circular attribute grammars
        \item $\square$   \alert{Circular remote attribute grammars}
    \end{description}
\end{frame}

\note[itemize]{
\item Next, we are going to describe circular remote attribute grammars
}


\begin{frame}{Circular Remote Attribute Grammar}{Definition: Circular + Remote AG}

\begin{definition}
A circular remote attribute grammar is an \alert{extension of remote attribute grammars} and has the same form as remote attribute grammars, except \alert{certain attributes are circular} and \alert{uses of attribute occurrences in functions are declared monotone in some arguments}.
\end{definition}
\end{frame}

\note[itemize]{
\item This is the informal definition of circular remote attribute grammar or CRAG
\item It is a merge of two previous two extensions: circular and remote into one
\item This is the most accommodating extension to describe circularly defined problems over a tree
}

% TODO
\begin{frame}{Circular Remote Attribute Grammar}{Applications}
Applications for CRAG:

\begin{itemize}
    \item First and Follow sets
    \item Unification (in type inference)
	\item Sub-typing with circular types (a class extending its own generic parameter)
	\item Detecting circular class extension in COOL semantic analyzer (e.g. A extends B, B extends C, C extends A)
 \item Program control-flow analysis
\end{itemize}
\end{frame}

\note[itemize]{
\item There are many applications for a circular remote attribute grammar including First and Follow in parsing, unification in type inference, detecting circularly defined class hierarchy during type checking, and control flow analysis
% \item COOL is a programming language (classroom object-oriented language)
}


\begin{frame}[fragile=singleslide]{Circular Remote Attribute Grammar}{Evaluation strategy}
Hedin in \cite{10.1016/j.scico.2005.06.005} used fixed-point loops with \alert{demand evaluation} to evaluate circular remote attribute grammar.

\begin{Verbatim}[fontsize=\small]
    do {
    
        // Evaluate rules (again?)
        
    } while (currentValue \supset prevValue)
\end{Verbatim}

\newlinevspace

But, what about \alert{schedule}? how to formalize the validity of the schedule for CRAG?
\end{frame}

\note[itemize]{
\item In CRAG, similar to circular attribute grammars we need a fixed point loop for evaluation
\item This is the basic setup of a fixed-point loop introduced by Hedin
\item Basically, we hold on to the previous value and then run the evaluation and then check whether the new value is a superset, NOT superset-equal to the previous value. And if that is the case then we re-run the loop.
}



\begin{frame}{Circular Remote Attribute Grammar}{Evaluation Terminating? Monotonicity}

\begin{block}{Remark}
The partial field write rule as defined in the definition of remote attribute grammar is a \alert{monotone} use.
\end{block}

\end{frame}

\note[itemize]{
\item Adding a monotonicity constraint ensures that fixed-point evaluation terminates
\item However, partial field writes are monotone, this is because object fields are \enquote{set} so they can be involved in a cycle
\item The only thing remaining is ensuring that arguments of semantic functions are monotone in a classical rule
}






\begin{frame}{Circular Remote Attribute Grammar}{Schedule Definition}
    
\begin{definition}
A schedule for a circular remote attribute grammar is a \alert{\customorder{}} $\lesssim$ on the \alert{instantiated rules}.
\end{definition}


Schedule is valid when ($a < b$ is $a \lesssim b \wedge b \not \lesssim a$):

% TODO: the same as 38
\[\begin{cases}
      \hat{r_i} < \hat{r_j},    & \text{whenever we have a simple dependency} \\
      \hat{r_i} \lesssim \hat{r_j}, & \text{whenever we have a monotone dependency}
    \end{cases}\]

    And for any two rules $\hat{r}_i = (v \texttt{=} w.f)$, $\hat{r}_j = (w'.f \sqsupseteq v' ) \in R(t)$ if these rules are \alert{potentially} referring to the same object then $\hat{r}_j \lesssim \hat{r}_i$ 

\end{frame}

\note[itemize]{
\item Schedule for CRAG is a \customorder{} of instantiated rules
\item Schedule is valid when dependencies defined using $\mathit{DO}$ and $\mathit{UO}$ are respected meaning that the definition of attribute instance precedes the use of the same attribute instance
\item And similarly writes to field of an object should precede the read of the field of the same object
}



% keep this :)
\begin{frame}[fragile=singleslide]{Example}{CRAG Example}

\begin{multicols}{3}
\begin{Verbatim}[fontsize=\small]
S -> A B
    local l
    l = A.r
    B.i = l
    A.i = B.s
    S.x = l.f
A -> a
    object o
    o.f <- A.i
    A.r = o


B -> b
    local l
    l = B.i
    B.s = l.f
\end{Verbatim}
\end{multicols}

\newlinevspace

$\Rightarrow$ Notice the \alert{cycle} involving reading and writing of the same object

\end{frame}

\note[itemize]{
\item This is an example of circular remote attribute grammar
\item Notice $o.f$ gets $A.i$ using partial write and $A.i$ itself gets $B.s$ using classical rule
\item Then $B.s$ gets $l.f$ where $l$ is a local attribute assigned
\item Basically, we write to $o.f$ using the value of $o.f$. Hence a cycle.
}


% kill this
\begin{frame}[fragile=singleslide]{Example}{Instantiated CRAG with trivial derivation}

\begin{multicols}{3}
\begin{Verbatim}[fontsize=\small]
n0: S -> A B
    local l0
    r0: l0 = n1.r
    r1: n2.i = l0
    r2: n1.i = n2.s
    r3: n0.x = l0.f
n1: A -> a
    object o0
    r4: o0.f <- n1.i
    r5: n1.r = o0


n2: B -> b
    local l1
    r6: l1 = n2.i
    r7: n2.s = l1.f
\end{Verbatim}
\end{multicols}

Schedule: 

\[
     \Big \{ \{ \hat{r}_5 \} < \{ \hat{r}_0 \} < \{ \hat{r}_1 \} < \{ \hat{r}_6 \} < \{ \hat{r}_4 , \hat{r}_7, \hat{r}_2 \} < \{ \hat{r}_3 \}  \Big \}
\]

\end{frame}

\note[itemize]{
\item This is the instantiated form of the circular remote attribute grammar we introduced in the previous slide
\item The corresponding schedule is at the bottom
\item Remember that \customorder{} is a total-order on a partition, hence why we used less-than $<$ here
}




% kill this
\begin{frame}[fragile=singleslide]{Example}{CRAG Example with non-monotone function $h$}

\begin{multicols}{3}
\begin{Verbatim}[fontsize=\small]
n0: S -> A B
    local l
    r0: l = h(n1.r)
    r1: n2.i = h(l)
    r2: n1.i = n2.s
    r3: n0.x = h(l.f)
n1: A -> a
    object o
    r4: o.f <- n1.i
    r5: n1.r = h(o0)


n2: B -> b
    local l
    r6: l = h(n2.i)
    r7: n2.s = l.f
\end{Verbatim}
\end{multicols}

\newlinevspace

The \textbf{trivial schedule} \alert{does not work} for this example because of $h$

\end{frame}

\note[itemize]{
\item If we introduce non-monotone function $h$ in a cycle, this means that the trivial schedule is no longer applicable or valid
}


\begin{frame}{$l$-Ordered Circular Remote Attribute Grammars}{Extending definition of $l$-ordered to circular remote AGs (CRAG)}
    
    \begin{lemma}\label{crag-lordered-wellformed}
$l$-ordered circular remote attribute grammars are well-formed.
\end{lemma}
\newlinevspace
    Proof by applying \enquote{fiber construction} and \enquote{fiber approximation} to proof of $l$-ordered circular attribute grammars
    
\end{frame}

\note[itemize]{
\item Lastly, we showed in the thesis that definition of $l$-ordered can be extended to circular remote attribute grammars thanks to \enquote{fiber-construction} and \enquote{fiber-approximation}
}
