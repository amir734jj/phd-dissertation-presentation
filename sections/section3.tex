
\section{Definitions}

\subsection*{Preliminaries}

\begin{frame}{Context-Free Grammar}{Foundation that attribute grammars built upon}

\begin{definition}
	Formally context-free grammar is a tuple of $G=(T, N, P, Z)$ where
	
	\begin{enumerate}
		\item $T$ is a set of terminal symbols
		\item $N$ is a set of non-terminal symbols
		\item $P$ is a set of production rules
		\item $Z$ is a start non-terminal symbol
	\end{enumerate}

\end{definition}
\end{frame}


\begin{frame}{Classical Attribute Grammar}{CFG + Rules}
\begin{definition}\label{def:ag-definition}
	A classical attribute grammar is a tuple $A = (G, S, I, L, R)$ where

	\begin{enumerate}
	    \item $G=(T, N, P, Z)$ is a context-free grammar
		\item $S$ is a set of \enquote{synthesized attributes} for each non-terminal $X$
		\item $I$ is a set of \enquote{inherited attributes} for each non-terminal $X$
		\item $L$ is a set of \enquote{local attributes} for each production $p$ 
		\item $R$ is a set of semantic rules of the form \alert{$v_0 \texttt{=} g(v_1, \dots, v_k)$} for each production $p$, where $g$ is a function, and $v_i \in V(p)$ are \enquote{attribute occurrences} of either the form $X.a$  or $l$
	\end{enumerate}

\end{definition}
\end{frame}

\begin{frame}{Notation}{Basic form of semantic rule used in classical AG}

Forms of semantic rules ($v_i$ are either $X.a$ or $l$):

\begin{enumerate}
    \item Classical rule: $$v_0 \texttt{=} g(v_1, \dots, v_k) $$
\end{enumerate}

\newlinevspace 
\newlinevspace 

\begin{block}{Shorthand}
If $g$ is the \alert{identity function}, then the rule can be written \alert{$v_0 = v_1$}
\end{block}
    
\end{frame}



\begin{frame}{Constraints}{Logical constraints to filter out AGs that do not provide any value}

\begin{definition}
An attribute grammar is \alert{well-formed} if there is exactly one defining semantic rule for each attribute given a derivation.
\end{definition}

\begin{definition}
An attribute grammar is \alert{well-defined} if it is \alert{well-formed}, and there is a unique solution that can be computed for each attribute given a derivation.
\end{definition}
\end{frame}




\begin{frame}{Defined and Used Occurrence Sets}{Divide occurrences into two disjoint sets}

In a production $p: X_0 \rightarrow X_1 \dots X_n$

\begin{equation}
\begin{gathered}
  \mathit{DO}(p) = \left \{ X_j.a \; \middle\vert
    \begin{array}{cc}
     a \in S(X_0) & j = 0 \\
     a \in I(X_k) & j \neq 0
    \end{array}
\right \} \cup L(p) \\
\mathit{UO}(p) = \left \{ X_j.a \; \middle\vert
    \begin{array}{cc}
     a \in I(X_0) & j = 0 \\
     a \in S(X_k) & j \neq 0
    \end{array}
\right \} \cup L(p) \\
\mathit{AO}(p) = \mathit{DO}(p) \cup \mathit{UO}(p)
\end{gathered} 
\end{equation}

\newlinevspace

$\to$ This will be used later to define \emph{compatibility} with $R(p)$

\end{frame}

\begin{frame}[fragile=singleslide]{Example}{Finding DO and UO sets for simple AG}

\begin{multicols}{2}
\begin{Verbatim}[fontsize=\small]
// DO = { A.i, B.i, S.s }
// UO = { A.s, B.s }
S -> A B
    S.s = A.s + B.s
    A.i = 'a'
    B.i = 'b'



// DO = { A.s }
// UO = { A.i }
A -> a
    A.s = A.i

// DO = { B.s }
// UO = { B.i }
B -> b
    B.s = B.i
\end{Verbatim}
\end{multicols}

\end{frame}


\begin{frame}{Derivation}{Embedding tree node in a derivation}
Basic form of a derivation $\mathscr{D}$
\[ \alpha_0 \rightarrow \alert{\alpha_1} \rightarrow \dots \rightarrow \alpha_n   \]

Example of a derivation $\mathscr{D}$
\[ S \rightarrow \alert{A B} \rightarrow a B \rightarrow a b \]

Example of a derivation $\hat{\mathscr{D}}$
\[ (S, 0) \rightarrow \alert{(A, 1) B} \rightarrow a (B, 2) \rightarrow a b  \]
\end{frame}


\begin{frame}{Instantiated Attribute Grammar}{Attribute Grammar + Derivation}
\begin{definition}
An instantiated attribute grammar is $\mathit{IA} = (A, \hat{\mathscr{D}}, \hat R)$ such that
\begin{enumerate}
	\item $A = (G, S, I, L, R)$ is an attribute grammar where  $G= (T, N, P, Z)$.
	
	\item Derivation \alert{$\hat{\mathscr{D}}$} or set of transitions. $(X, i)$ is called instantiated node.
    
    \item \alert{$\hat R$} is a set of instantiated semantic rules, defined as follows: for each step of derivation $\alpha_i \rightarrow \alpha_{i+1}$ which has a form of $\hat{\alpha}(X_0,j)\hat{\beta} \rightarrow \hat{\alpha}\hat{\delta}\hat{\beta}$ we have an associated production $p: X_0 \rightarrow X_1 \dots X_n$ that made that derivation step possible.
\end{enumerate}
\end{definition}

\end{frame}


\begin{frame}{Example}{Instantiated rule}
Rule in production $S \rightarrow A B$
\[ r: S.s = A.s + B.s  \]

Instantiated rule: occurrences replaced with \alert{attribute instanced}
\[ \hat{r_0}: (S, 0).a = (A, 1).s + (B,2).s  \]
\end{frame}

\begin{frame}{Example}{Instantiated attribute grammar rules $\hat{R}$}


{\small
$S \rightarrow A B$
\begin{gather*}
\hat{r_0}: (S, 0).a = (A, 1).s + (B,2).s \\
\hat{r_1}: (A, 1).i = `a` \\
\hat{r_2}: (B, 2).i = `b`
\end{gather*}

$A \rightarrow a$
\begin{gather*}
\hat{r_3}: (A,1).s = (A,1).i \\
\end{gather*}

$B \rightarrow b$
\begin{gather*}
\hat{r_4}: (B,2).s = (B,2).i \\
\end{gather*}
}



\end{frame}


\begin{frame}{Symbols}{Useful symbols for time complexity analysis}

Set of \alert{all tree nodes} and set of \alert{all attribute instances}

\begin{equation}
\begin{gathered}
    \hat{N} = \bigcup_{\hat r \in \hat R} \{ (X, i) \mid \hat v_i \in \hat r \wedge \hat v_i \equiv (X, i).a \}  \\
    \hat{V} = \bigcup_{\hat r \in \hat R} \begin{cases}
(X, i).a &  \hat v_i \in \hat r \wedge \hat v_i \equiv (X, i).a \\
(l, i)   &  \hat v_i \in \hat r \wedge \hat v_i \equiv (l, i) 
\end{cases}
\end{gathered}
\end{equation}
\end{frame}


\subsection*{Evaluations}

\begin{frame}{Demand Evaluation}{Straightforward evaluation method}
\begin{definition}
Demand evaluation is a kind of evaluation where each attribute instance access requires a call to evaluate the corresponding instantiated semantic rule.
\end{definition}

\begin{gather*}
\alert{(S, 0).a } \to \hat{r_0}: (S, 0).a = \alert{(A, 1).s} + \alert{(B,2).s} \\
\to (A, 1).s \to \hat{r_3}: (A,1).s = \alert{(A,1).i} \\
\to (B,2).s \to \hat{r_4}: (B,2).s = \alert{(B,2).i} \\
\to (A,1).i \to \hat{r_1}: (A, 1).i = `a` \\
\to (B,2).i \to \hat{r_2}: (B, 2).i = `b`
\end{gather*}
\end{frame}


\begin{frame}{Demand Evaluation}{Pros and cons}
Pros:
\begin{itemize}
    \item Can benefit form \alert{caching to prevent re-evaluation}
    \item Easy to implement
\end{itemize}

Cons:
\begin{itemize}
    \item Does not detect cycles before evaluation begins and \alert{may not terminate}
    \item Space complexity if \alert{caching} is used: $\mathcal{O}(|\hat{V}|)$
    \item Time complexity of: $\mathcal{O}(| \hat{V} | \times | \hat{R} |)$ (for classical AG)
\end{itemize}
\end{frame}


\begin{frame}{Schedule Evaluation}{Easy to implement}

Given this \emph{schedule}, evaluate the AG:

\[ \mathit{schedule} = \{ \hat{r_1} < \hat{r_2} < \hat{r_3} < \hat{r_4} < \hat{r_0}  \} \]

\begin{alertblock}{Observation}
Finding the schedule requires topological sort of attribute instances: $\mathcal{O}(| V +  E|)$ where $V = \hat{V}$ and $E = |\hat{V}  \times  \hat{R}|$.
\end{alertblock}
\end{frame}


\begin{frame}{Schedule Evaluation}{Pros and cons}
Pros:
\begin{itemize}
    \item If scheduler finds a schedule, then evaluation \alert{will terminate}
    \item Easy to implement
    \item \alert{No re-evaluation}
    \item Better space complexity as caching is not used
    \item Time complexity of: $\mathcal{O}(| \hat{R} |)$ (for classical AG)
\end{itemize}

Cons:
\begin{itemize}
    \item Schedule \alert{works only for one derivation} of attribute grammar
\end{itemize}
\end{frame}


\begin{frame}{Schedule}{Total order of instantiated rules}
What makes a schedule (total order on instantiated rules) valid?

$\to$ \alert{No use of attribute instance before it is defined first}

\newlinevspace

\begin{definition}
For all $\hat{r_1}, \hat{r_2} \in \hat R$ then  $\hat{r_1} < \hat{r_2}$ iff $\exists \hat{v}_i \in \hat{r_1} (\exists \hat{v}_i \in \hat{r_2} (\texttt{LHS($\hat{r_1}$)} = \hat{v}_i \wedge \hat{v}_i \in \texttt{RHS($\hat{r_2}$)} ))$, meaning that rule $\hat{r_1}$ or simply defines $\hat{v}_i$ and rule $\hat{r_2}$ uses $\hat{v}_i$.
\end{definition}
\end{frame}


\begin{frame}[fragile=singleslide]{Example}{Example of AG that can produce infinite number of derivations}

Finding the schedule for a particular derivation of this AG is much more harder as a derivation may have many many nodes. 

\begin{multicols}{3}
\begin{Verbatim}[fontsize=\small]
S -> A
  A.i1  = S.in
  A.i2  = A.s1
  S.out = A.s2




A -> A A
  A1.i1 = A0.i1
  A1.i2 = A1.s1
  A0.s1 = A1.s2
  A2.i1 = A0.i2
  A2.i2 = A2.s1
  A0.s2 = A2.s2

A -> epsilon
  A.s1 = A.i1 + 1
  A.s2 = A.i2 + 2
    
    
    
    
\end{Verbatim}
\end{multicols}

\end{frame}




\begin{frame}{Static Evaluation}{Alternative evaluation method: Constructed independent of a derivation}

\begin{itemize}
    \item What if there was a way to \alert{statically} evaluate attribute grammars?
    \begin{itemize}
        \item Generate evaluator \alert{once} and would \alert{work for all possible derivations} of an attribute grammar.
    \end{itemize}
    \item What \alert{constraints} need to be true for such an attribute grammar to have a static evaluator?
    \item Does evaluation \alert{terminate}?
    \item How to verify if static evaluation is \alert{valid}?
\end{itemize}
\end{frame}



\begin{frame}{Visit Sequence Evaluation}{Static evaluator}

$\to$ \alert{Visit sequence evaluation} is a \alert{type of static evaluation}.

\newlinevspace

Pros:
\begin{itemize}
    \item Evaluation \alert{will terminate}
    \item Easy to implement
    \item \alert{No re-evaluation}
    \item Better space complexity as caching is not used
    \item Time complexity of: $\mathcal{O}(| \hat{R} |)$ (for classical AG)
    \item \alert{Practical}: can be generated independent of derivation
\end{itemize}

Cons:
\begin{itemize}
    \item Only works on \alert{$l$-ordered} class of attribute grammars
\end{itemize}
\end{frame}



\begin{frame}[fragile=singleslide]{Example}{Visit sequence evaluator}

$\to$ Set of \alert{recursive functions} where each function takes an \alert{instance of non-terminal} (derivation tree node)

\begin{multicols}{2}
\begin{Verbatim}[fontsize=\scriptsize]
visit_S(n: S -> A)
    A.i1 = S.in
    - visit_A_part1(A)
    A.i2 = A.s1
    - visit_A_part1(A)
    S.out = A.s2

visit_A_part1(n: A -> A A)
    A1.i1 = A0.i1
    - visit_A_part2(A1)
    A1.i2 = A1.s1
    - visit_A_part3(A1)
    A0.s1 = A1.s2
visit_A_part2(n: A -> A A)
    A2.i1 = A0.i2
    - visit_A_part2(A2)
    A2.i2 = A2.s1
    - visit_A_part3(A2)
    A0.s2 = A2.s2

visit_A_part1(n: A -> epsilon)
    A.s1 = A.i1 + 1

visit_A_part2(n: A -> epsilon)
    A.s2 = A.i2 + 1
    
\end{Verbatim}
\end{multicols}
\end{frame}


\begin{frame}{$l$-ORD}{$T(X)$: Total order of attributes for non-terminal $X$ }

\[  T(S) = \{ S.\mathit{in} < S.\mathit{out}  \}  \]

\[  T(A) = \{ A.i_1 < A.i_2 < A.s_1 < A.s_2 \}  \]

\newlinevspace

$\to$ AG is $l$-ordered if exists a $T(X)$  for all non-terminal $X \in N$ such that its \alert{compatible} with $\mathit{AO}(p)$ and $R(p)$ where $p: X \rightarrow \alpha$. 

\end{frame}

\begin{frame}{$l$-ORD}{Definition}

\begin{definition}
An attribute grammar is called $l$-ordered if there exists a function $T$ that maps each symbol $X$ to a total order $T(X) \subseteq (\mathit{Attr}(X) \times \mathit{Attr}(X))$ of attributes of $X$ that is \alert{compatible} with the whole attribute grammar. That is $\forall p \in P$ the function $T$ is \alert{compatible} with the total order of the attribute occurrences $\mathit{Ord}(\mathit{AO}(p)) \subseteq (V(p) \times V(p))$ and total order of semantic rules $R(p)$.

\end{definition}
\end{frame}

\begin{frame}{$l$-ORD}{Compatibility}
    \begin{enumerate}
        \item \alert{$T$} and $\mathit{Ord}(\alert{\mathit{AO}(p)})$ are called compatible if for any two occurrence of the same non-terminal $(X, i).a, (X, i).a' \in \mathit{AO}(p)$ then $(X.a < X.a') \in \mathit{Ord}(\mathit{AO}(p))$ iff $(a < a') \in T(X)$.

        \item $\mathit{Ord}(\alert{\mathit{AO}(p)})$ and \alert{$R(p)$} are called compatible if for any two occurrence $X.a$, $\allowbreak X'.a' \allowbreak \in \mathit{AO}(p)$ then $(X.a < X'.a') \in \mathit{Ord}(\mathit{AO}(p))$ iff $X.a \in \mathit{DO}(p)$ and $X'.a'  \allowbreak \in \mathit{UO}(p)$.
    \end{enumerate}

\end{frame}

\begin{frame}{Visit Sequence Evaluation}{Protocol}

\begin{definition}
A \emph{Protocol} $\Pi(X)$ is an \alert{ordered partition of attributes} for each non-terminal $X \in N$, where it can include an empty set or multiple synthesized or inherited attributes.
\end{definition}

\newlinevspace

\begin{examples}
For example, $\Pi(A)$ is the following:
\[ \Pi(A) = \Big\{ \{ A.i_1, A.s_1 \} < \{ A.i_2, A.s_2 \} \Big\} \]
\end{examples}
\end{frame}


\begin{frame}{Visit Sequence Evaluation}{Visit sequence $\mathscr{V}(p)$}

\begin{definition}
A \emph{visit-sequence} for a production $p: X \rightarrow \alpha \in P$ of a $l$-ordered attribute grammar is $\mathscr{V}(p)$, a sequence of visits or a \alert{two-level ordered partitions} of rules where each visit (or outer set) corresponds to $\Pi(X)$. Visit sequence is \alert{compatible} with $R(p)$. In other words, for any two attribute occurrence $v_i$ and $v_i'$ defined or used in the rules $r$ and $r'$ respectively which themselves are a part of $\mathscr{V}(p)$, then $(r < r') \in \mathscr{V}(p)$ iff $v_i \in \mathit{DO}(p)$ and $v_j \in \mathit{UO}(p)$.
\end{definition}

\end{frame}

\begin{frame}{Example}{Visit sequence for a production $\mathscr{V}(p: A \rightarrow A A)$}

\[ \Pi(A) = \Big\{ \{ A.i_1, A.s_1 \} < \{ A.i_2, A.s_2 \} \Big\} \]

\begin{equation}
\begin{gathered}
\begin{split}
\mathscr{V}(p: A_0 \rightarrow A_1 A_2) = \Bigg\{               
   \overbrace{ \Big\{  \mathit{visit}(A_1, 1)  < \mathit{visit}(A_1, 2)	 \Big\}}^\text{\alert{first visit} corresponding to first set in $\Pi(A)$ }  <  \\
   \underbrace{ \Big\{  \mathit{visit}(A_2, 1)  < \mathit{visit}(A_2, 2) \Big\}}_\text{\alert{second visit} corresponding to second set in $\Pi(A)$}
\Bigg\}
\end{split}
\end{gathered}
\end{equation}

\end{frame}


\begin{frame}{Visit Sequence Evaluator}{Interpreter style program that used visit sequence}

\begin{definition}
Visit sequence evaluator is a static attribute grammar evaluator that is a \alert{recursively defined function} and takes as argument: a production $p: X \rightarrow \alpha$, an instance of non-terminal node of LHS of production $(X:i, j)$, \alert{a visit sequence $\mathscr{V}$} and \alert{derivation $\hat{\mathscr{D}}$} and uses them to evaluate all the attribute occurrences \alert{$\mathit{AO}(p)$}.
\end{definition}

\end{frame}


\begin{frame}{Visit Sequence Evaluator}{Recursive function}

\begin{center}
    \scalebox{0.5}{
    \begin{minipage}{1.7\linewidth}
\begin{algorithm}[H]
\begin{algorithmic}[1]
\Procedure{\texttt{VISIT\_SEQUENCE\_EVAL}}{$\mathit{AG}, \hat{\mathscr{D}}, \; (X:i, j), \; k, \; \mathit{Val}$}
    \State{$(p: X \rightarrow \alpha, \; \alpha_l \rightarrow \alpha_{l+1}) \gets \hat{\mathscr{D}}[j]$}
    \For{$(V', m) \gets (\mathscr{V}(p) \times \mathbb{N})$}
        \If{$m = k$} \Comment{Evaluate only visit number $k$}
            \For{$\mathscr{I} \gets V'$}
                \If{$\mathscr{I} \equiv visit(X':n, o)$}
                    \State \Comment{Find rules that define child inherited attributes if any}
                    \State{$\texttt{EVAL\_TREE\_ATTRS}(\mathit{AG}, p, (X':n), \mathit{Val}    )$}
                    \State
                    \State \AlgMultiline{$q \gets \exists q \in [j + 1 , \dots,  |\hat{\mathscr{D}}|]  \; ( \alpha_q \rightarrow \alpha_{q+1} \equiv \hat{\mathscr{D}}[q] \wedge \allowbreak  \alpha_q \equiv \hat{\alpha}(X':n,q)\hat{\beta} \wedge  \alpha_{q+1} \equiv  \hat{\alpha}\hat{\delta}\hat{\beta} )$}
                    \State{$\texttt{VISIT\_SEQUENCE\_EVAL}(\mathit{AG}, \hat{\mathscr{D}}, \; (X':n, q), \; o, \; \mathit{Val}    )$}
                \ElsIf{$\mathscr{I} \equiv r$}
                    \State{$\hat{r} \gets \sigma(r)$}
                    \State{$\texttt{AG\_EVAL}(\hat{r}, \mathit{Val} )$}
                \EndIf
            \EndFor
            \State \Comment{Find rules that define parent synthesized attributes if any}
            \State{$\texttt{EVAL\_TREE\_ATTRS}(\mathit{AG}, p, (X:i), \mathit{Val}    )$}
        \EndIf
    \EndFor
\EndProcedure
\end{algorithmic}
\caption{Schedule evaluation of circular remote attribute grammar}
\end{algorithm}
    \end{minipage}%
    }    
\end{center}
    
    
\end{frame}


\begin{frame}{Visualization Tools}{Dependency graph}

\begin{figure}[htbp]
    \centering
    \resizebox{.8\textwidth}{!}{
    \begin{tikzpicture}[->,>=Stealth,auto,
        thick,main node/.style={draw, rectangle, align=center, text width=1cm}]

\node[main node, ] (1) at (-2,4)   {$A_0.i_1$};
\node[main node, anchor=north west] (2) at(1.north east) {$A_0.i_2$};
\node[main node, anchor=south west] (3) at(2.north east) {$A_0.s_1$};
\node[main node, anchor=north west] (4) at(3.north east) {$A_0.s_2$};


\node[main node, ] (5) at (-6,0)   {$A_1.i_1$};
\node[main node, anchor=north west] (6) at(5.north east) {$A_1.i_2$};
\node[main node, anchor=south west] (7) at(6.north east) {$A_1.s_1$};
\node[main node, anchor=north west] (8) at(7.north east) {$A_1.s_2$};

\node[main node, ] (9) at (2,0)   {$A_2.i_1$};
\node[main node, anchor=north west] (10) at(9.north east) {$A_2.i_2$};
\node[main node, anchor=south west] (11) at(10.north east) {$A_2.s_1$};
\node[main node, anchor=north west] (12) at(11.north east) {$A_2.s_2$};

\draw (12.north) to[out=90, in=-90, looseness=1] (4.south);
\draw (2.south) to[out=-90, in=90, looseness=1] (9.north);
\draw (8.north) to[out=90, in=-90, looseness=1] (3.south);
\draw (1.south) to[out=-90, in=90, looseness=1] (5.north);

\draw (11.north) to[out=90, in=90, looseness=1.5] (10.north);
\draw (7.north) to[out=90, in=90, looseness=1.5] (6.north);


\end{tikzpicture}}

    \caption{Dependency graph $\mathit{DG}_{p: A \rightarrow A A}$}
\end{figure}

\end{frame}

\begin{frame}{Visualization Tools}{Augmented dependency graph}

\begin{figure}[htbp]
    \centering

    \resizebox{.8\textwidth}{!}{
    \begin{tikzpicture}[->,>=Stealth,auto,
        thick,main node/.style={draw, rectangle, align=center, text width=1cm}]

\node[main node, ] (1) at (-2,4)   {$A_0.i_1$};
\node[main node, anchor=north west] (2) at(1.north east) {$A_0.i_2$};
\node[main node, anchor=south west] (3) at(2.north east) {$A_0.s_1$};
\node[main node, anchor=north west] (4) at(3.north east) {$A_0.s_2$};


\node[main node, ] (5) at (-6,0)   {$A_1.i_1$};
\node[main node, anchor=north west] (6) at(5.north east) {$A_1.i_2$};
\node[main node, anchor=south west] (7) at(6.north east) {$A_1.s_1$};
\node[main node, anchor=north west] (8) at(7.north east) {$A_1.s_2$};

\node[main node, ] (9) at (2,0)   {$A_2.i_1$};
\node[main node, anchor=north west] (10) at(9.north east) {$A_2.i_2$};
\node[main node, anchor=south west] (11) at(10.north east) {$A_2.s_1$};
\node[main node, anchor=north west] (12) at(11.north east) {$A_2.s_2$};

\draw (12.north) to[out=90, in=-90, looseness=1] (4.south);
\draw (2.south) to[out=-90, in=90, looseness=1] (9.north);
\draw (8.north) to[out=90, in=-90, looseness=1] (3.south);
\draw (1.south) to[out=-90, in=90, looseness=1] (5.north);

\draw (11.north) to[out=90, in=90, looseness=1.5] (10.north);
\draw (7.north) to[out=90, in=90, looseness=1.5] (6.north);

\draw[red] (5.south) to[out=-90, in=-90, looseness=1] (7.south);
\draw[red] (6.south) to[out=-90, in=-90, looseness=1] (8.south);
\draw[red] (9.south) to[out=-90, in=-90, looseness=1] (11.south);
\draw[red] (10.south) to[out=-90, in=-90, looseness=1] (12.south);

\end{tikzpicture} }

    \caption{Augmented dependency graph $\mathit{DG}_{p: A \rightarrow A A}^*$}
\end{figure}

\end{frame}


\begin{frame}{Ordered AG Test}{Test if visit sequence evaluator exists (efficiently)}
By definition, there exist a visit sequence for \alert{$l$-ordered} attribute grammars but membership test is \alert{NP-complete} \cite{ENGELFRIET1982283} so its common/practical to use an OAG test which is a greedy algorithm that runs in \alert{polynomial time}.

\[ \mathit{OAG} \subseteq \mathit{l\text{-ordered}} \]

\end{frame}

\begin{frame}{Ordered Attribute Grammar (OAG)}{Definition}

\begin{definition}
An attribute grammar is an Ordered Attribute Grammar (OAG) if for every production $p \in P$, the graph $\mathit{DG}_p^*$ is \alert{cycle free}.
\end{definition}

\begin{block}{Remark}
OAG test runs in polynomial time or $\mathcal{O}(|R|)$
\end{block}


\end{frame}


\subsection*{Extensions}

\begin{frame}{Circular Attribute Grammar}{Classical AG + allowing circularity}
\begin{definition}
An attribute grammar is \alert{circular} iff there exists a derivation tree of the context-free grammar whose \alert{attribute dependency has a cycle}. Conversely, an attribute grammar is non-circular if there is a valid schedule for every possible derivation tree.
\end{definition}
\end{frame}


\begin{frame}{Circular AG Evaluation}{Ascending chain condition}

Is it possible to evaluate circular attribute grammars? what if \alert{evaluation never terminates}?

\newlinevspace

$\to$ Require that the domain of all attributes involved in cyclic chains can be arranged in a \alert{lattice of finite height} and that all semantic functions for these attributes are \alert{monotonic}

\[ x_{i+1} = f(x_i) \text{ where } x_0 = \bot \]

\end{frame}

\begin{frame}{Circular AG Evaluation}{Synth functions}
\alert{Synth function evaluators} for circular AG was introduced by Farrow in \cite{10.1145/13310.13320}

\newlinevspace

{ \footnotesize 

Pros:
\begin{itemize}
    \item Type of \alert{static evaluation}
    \item Easy to implement
    \item Uses \alert{fixed-point loop} to ensure fixed-point in attribute values
\end{itemize}

Cons:
\begin{itemize}
    \item Allows \alert{nested loops}, that is if there is another cycle during the iterated evaluation of circularly defined attribute values. Number of iteration of the innermost loop becomes an \alert{exponential factor} of the nested level of the loop in the worst case.
\end{itemize} }

\end{frame}


\begin{frame}{Example}{Synth function to calculate $s_1$ and $s_2$ attributes}

\begin{minipage}{0.8\linewidth}
\scriptsize
\begin{algorithmic}
\Function{$\texttt{EVAL\_A\_s1}$}{$n$}
    \If{$\texttt{PRODUCTION\_OF($n$)} = p: A_0 \rightarrow A_1 A_2$}
        \State \Return{$A_1.s_2$}
    \ElsIf{$\texttt{PRODUCTION\_OF($n$)} =  p: A \rightarrow \epsilon$}
        \State \Return{$n.i_1 + 1$}
    \EndIf
\EndFunction
\State
\Function{$\texttt{EVAL\_A\_s2}$}{$n$}
    \If{$\texttt{PRODUCTION\_OF($n$)} = p: A_0 \rightarrow A_1 A_2$}
        \State \Return{$A_2.s_2$}
    \ElsIf{$\texttt{PRODUCTION\_OF($n$)} =  p: A \rightarrow \epsilon$}
        \State \Return{$n.i_2 + 2$}
    \EndIf
\EndFunction
\end{algorithmic}
\end{minipage}

\end{frame}


\begin{frame}{Remote Attribute Grammar}{Classical AG + allowing references}
\begin{definition}
A remote attribute grammar is a tuple $(G,S,I,L,R,\alert{B},\alert{F})$ where \alert{$B$ is a set of objects} declared at each production and \alert{$F$ is a set of fields} that each object has.
\end{definition}

\newlinevspace

In RAG, we can access attribute values that are defined \alert{non-locally} whereas in classical all attributes are defined \alert{locally}.

\end{frame}



\begin{frame}{Notation}{Basic form of semantic rule uses in remote AG}

Forms of semantic rules:

\begin{enumerate}
    \item Classical rule: $$v_0 \texttt{=} g(v_1, \dots, v_k) $$
    \item Field read: $$v_i = w.f$$
    \item Partial field write: $$w.f \sqsupseteq v_i$$
\end{enumerate}

\end{frame}


\begin{frame}{Observation}{Partial field write}

It is valid to have \alert{multiple partial field write} for the same field of an object. This is \alert{unlike classical rule} where we can define an attribute instance \alert{once}.

\newlinevspace

This makes it challenging as schedule for RAG should order instantiated rules such that \alert{read of the object's field} should be done when its value is \alert{final}. 

\end{frame}

\begin{frame}{Remote Attribute Grammar}{$\mathit{DO}$ and $\mathit{UO}$ sets for RAG}
In a production $p: X_0 \rightarrow X_1 \dots X_n$

\begin{equation}
\small
\begin{gathered}
\mathit{DO}(p) = \left \{ X_j.a \; \middle\vert
    \begin{array}{cc}
     a \in S(X_0) & j = 0 \\
     a \in I(X_k) & j \neq 0
    \end{array}
\right \} \cup L(p) \cup 
\{ O_0 \mid O_0 \in B(p)  \} \\
\mathit{UO}(p) = \left \{ X_j.a \; \middle\vert
    \begin{array}{cc}
     a \in I(X_0) & j = 0 \\
     a \in S(X_k) & j \neq 0
    \end{array}
\right \} \cup L(p) \cup \{ O_0 \mid O_0 \in B(p)  \}
\end{gathered}
\end{equation}
\end{frame}

\begin{frame}[fragile=singleslide]{Example}{A simple remote attribute grammar definition}

\begin{multicols}{3}
\begin{Verbatim}[fontsize=\small]
S -> A B
    B.i = A.s
    S.x = A.r


A -> a
    object o
    A.s = o
    A.r = o.f

B -> b
    local l
    l = B.i
    l.f <- 'b'

\end{Verbatim}
\end{multicols}
    
\end{frame}





\begin{frame}{Remote Attribute Grammar}{Object reference availability notation}

$\to$ Calculating $B$ set is used to \alert{verify schedule for RAG} \cite{Boyland05remoteattribute}

\newlinevspace

\begin{equation}
\small
  \begin{gathered}
\begin{cases}
B(v_0) \supseteq B(v_i)    & (v_0 \texttt{=} g(v_1, \dots, v_n )) \in \hat R \text{ when } L_{g i} = 0 \\
B(v) \supseteq B(v') &  (v = w. f ), (w'.f \sqsupseteq v') \in \hat R \text{ when } B(w) \cap B(w') \neq \emptyset \\
B(o) = \{o\} & o \in B(t) \\
\end{cases}
  \end{gathered}
\end{equation}

\end{frame}

\begin{frame}{Remote Attribute Grammar}{Schedule}

\begin{definition}
Schedule then for a remote attribute grammar is a strict total order ($<$) on the instantiated rules when the two conditions below are met.
 
\begin{minipage}[t]{\linegoal}
\begin{enumerate}

\item {[Same as schedule for classical attribute grammar]}

\item For two rules $r_1 = (v = w.f)$, $r_2 = (w'.f \sqsupseteq v')$ if these rules may be referring to the same object or $B(w) \cap B(w') \neq \emptyset$, then the partial field write must strictly precede the field read or $r_2 < r_1$.
\end{enumerate}
\end{minipage}
\end{definition}
    
\end{frame}


\begin{frame}[fragile=singleslide]{Example}{Instantiated remote attribute grammar + schedule + $B$ set calculation}

\begin{multicols}{3}
\begin{Verbatim}[fontsize=\small]
n0: S -> A B
    r0: n2.i = n1.s
    r1: n0.x = n1.r


n1: A -> a
    object o0
    r2: n1.s = o0
    r3: n1.r = o0.f
    
n2: B -> b
    r4: l0 = n2.i
    r5: l0.f <- 'b'

\end{Verbatim}
\end{multicols}

\begin{equation}\label{eq:b-set-rag}
  \begin{gathered}
B(l_0) \supseteq B(n_2.i) \supseteq B(n_1.s) \supseteq B(o_0) = \{ o_0 \} \\
B(n_0.x) \supseteq B(n_1.r) = \emptyset
  \end{gathered}
\end{equation}


\begin{equation}\label{eq:remote-ag-schedule}
   \Big \{ \hat{r}_2 < \hat{r}_0 < \hat{r}_4 < \hat{r}_5 < \hat{r}_3 < \hat{r}_1 \Big \}
\end{equation}


\end{frame}



\begin{frame}{Circular Remote Attribute Grammar}{Definition: Circular + Remote AG}

\begin{definition}
Circular remote attribute grammar is an \alert{extension of remote attribute grammars} and has the same form as remote attribute grammars, except \alert{certain attributes are circular} and \alert{some functions are declared monotone in some arguments}. This monotonicity is declared as $M_{gi} = \texttt{true}$, meaning that function $g$ is monotone in argument $i$.
\end{definition}
\end{frame}


\begin{frame}{Circular Remote Attribute Grammar}{Applications}
Obvious Applications for CRAG:

\begin{itemize}
    \item Unification (in type inference)
	\item Sub-typing with circular types (class extending its own generic parameter)
	\item Detecting circular class extension in Cool semantic analyzer (e.g. A extends B, B extends C, C extends A)
\end{itemize}
\end{frame}


\begin{frame}[fragile=singleslide]{Circular Remote Attribute Grammar}{Evaluation strategy}
Hedin in \cite{10.1016/j.scico.2005.06.005} used fixed-point loops with \alert{demand evaluation} to evaluate circular remote attribute grammar.

\begin{Verbatim}[fontsize=\small]
    do {
    
        // Evaluate rules (again?)
        
    } while (currentValue \supset prevValue)
\end{Verbatim}

\newlinevspace

But, what about \alert{schedule}? how to formalize validity of schedule for CRAG?
\end{frame}


\begin{frame}{Notation}{Monotone definition}

\begin{definition}
If $f:X \rightarrow Y$ is a \alert{set function} from a collection of sets $X$ to an ordered set $Y$, then $f$ is said to be monotone if whenever \alert{$A \subseteq B$} as elements of $X$, \alert{$f(A) \leq f(B)$}
\end{definition}

\end{frame}


\begin{frame}{Circular Remote Attribute Grammar}{Evaluation Terminating? Monotonicity}
$\to$ If we put instantiated rules in a fixed-point loop, How to \alert{ensure evaluation terminates}?

\newlinevspace

Solution: \alert{Monotonicity} of arguments in a classical rule

\newlinevspace

\begin{block}{Remark}
Partial field write rule as defined in definition of remote attribute grammar \alert{is a monotone operation}.
\end{block}
\end{frame}



\begin{frame}{Monotone Use}{Validate monotonicity in argument of classical rule}


\begin{definition}
Given classical attribute grammar rule $v_0 \texttt{=} g(v_1, \dots, v_j, \dots, v_n)$, $M_{g j} = \texttt{true}$ or $\texttt{false}$ indicates whether use of attribute instance at argument $j$ of function $g$ is monotone if it yields \texttt{true}. If it yields \texttt{false} then it is not non-monotone.
\end{definition}    
    
\end{frame}

\begin{frame}{Simple and Monotone Use}{Dividing the $\mathit{UO}$ set into two disjoint sets}

\begin{itemize}
    \item $\mathit{SUO}(p)$ is referring to \alert{simple} (non-monotone) use occurrences  
    \item $\mathit{MUO}(p)$ refers to \alert{monotone} use occurrences
\end{itemize}

\newlinevspace

    \[ \mathit{UO}(p) = \mathit{SUO}(p) \cup \mathit{MUO}(p) \]

\end{frame}

\begin{frame}{Notation}{Total pre-order}

Formally, a total pre-order relation $\lesssim$ satisfies the following properties:

\begin{enumerate}
    \item For all $x,y$ and $z$, if $x\lesssim y$ and $y\lesssim z$ then $x\lesssim z$ (\alert{transitivity}).
    \item For all $x, y$, then $x\lesssim y$ or $y\lesssim x$ must be true (strong connectedness or \alert{total}). Hence, for all $x$, then $x\lesssim x$ must also be true (\alert{reflexivity}).
\end{enumerate}    

\end{frame}

\begin{frame}{Circular Remote Attribute Grammar}{Schedule Definition part \#0}
    
    \begin{definition}
A schedule then for a circular remote attribute grammar is a \alert{total pre-order} $\lesssim$ on the \alert{set of instantiated rules} such that the following \alert{three conditions} are met.
\end{definition}
    
\end{frame}

\begin{frame}{Circular Remote Attribute Grammar}{Schedule Definition part \#1}
    
    \begin{definition}
\begin{enumerate}
\setcounter{enumi}{0}
  \item Every rule that uses some attribute instance must be scheduled after any rule that defines the same attribute instance. Formally, for all $\hat{r_1}, \hat{r_2} \in \hat R$ then $\hat{r_1} < \hat{r_2}$ or $\hat{r_1} \leq \hat{r_2}$ iff $\exists \hat{v}_i \in \hat{r_1} (\exists \hat{v}_i \in \hat{r_2} (\texttt{LHS($\hat{r_1}$)} = \hat{v}_i \wedge \hat{v}_i \in \texttt{RHS($\hat{r_2}$)}   \wedge \hat{v}_i \notin B(t) ))$. This means that rule $\hat{r_1}$ which defines $\hat{v}_i$ must precede rule $\hat{r_2}$ which uses $\hat{v}_i$. The following defines the condition of when $<$ or $\leq$ is used:

    \[ \begin{cases}
      \hat{r_1} < \hat{r_2},    & \text{if } \hat{v}_i \in \mathit{SUO}(p) \\
      \hat{r_1} \leq \hat{r_2}, & \text{otherwise (} \hat{v}_i \in \mathit{MUO}(p)  \text{)}     
    \end{cases} \]
\end{enumerate}
\end{definition}
    
\end{frame}

\begin{frame}{Circular Remote Attribute Grammar}{Schedule Definition part \#2}
\begin{definition}

\begin{enumerate}
\setcounter{enumi}{1}
  \item For two rules $r_1 = (v = w.f)$, $r_2 = (w'.f \supseteq v' ) \in R(t)$ if these rules are referring to the same object or $B(w) \cap B(w') \neq \emptyset$, then the partial field write must precede the field read or be done in the same set, meaning $r_2 \leq r_1$.
\end{enumerate}
\end{definition}
\end{frame}

\begin{frame}{Circular Remote Attribute Grammar}{Schedule Definition part \#3}
\begin{definition}
A schedule then for a circular remote attribute grammar is a total pre-order $\lesssim$ on the set of instantiated rules such that the following two conditions are met.

\begin{enumerate}
\setcounter{enumi}{2}
  \item For all $\hat{r}_0 = v_0 \texttt{=} g(v_1, \dots , v_k)$ and for all $\hat{r}_i$ defining $v_i$, that is $\texttt{LHS($r_i$)} = v_i$, where $i \in [1, k]$, the schedule must follow this relation:
  
  \[ 
  \begin{cases}
  \hat{r}_i \le \hat{r}_0   & \text{if } M_{gi} = \texttt{true} \\
  \hat{r}_i < \hat{r}_0 & \text{otherwise}
  \end{cases}
  \]
\end{enumerate}
\end{definition}
\end{frame}


\begin{frame}[fragile=singleslide]{Example}{CRAG Example}

\begin{multicols}{3}
\begin{Verbatim}[fontsize=\small]
S -> A B
    local l
    l = A.r
    B.i = l
    A.i = B.s
    S.x = l.f
A -> a
    object o
    o.f <- A.i
    A.r = o


B -> b
    local l
    l = B.i
    B.s = l.f
\end{Verbatim}
\end{multicols}

\newlinevspace

$\to$ Notice the \alert{cycle} involving reading and writing of the same object

\end{frame}


\begin{frame}[fragile=singleslide]{Example}{Instantiated CRAG with trivial derivation}

\begin{multicols}{3}
\begin{Verbatim}[fontsize=\small]
n0: S -> A B
    local l0
    r0: l0 = n1.r
    r1: n2.i = l0
    r2: n1.i = n2.s
    r3: n0.x = l0.f
n1: A -> a
    object o0
    r4: o0.f <- n1.i
    r5: n1.r = o0


n2: B -> b
    local l1
    r6: l1 = n2.i
    r7: n2.s = l1.f
\end{Verbatim}
\end{multicols}


\end{frame}

\begin{frame}[fragile=singleslide]{Example}{Schedule + $B$ set calculation}


\begin{equation}
  \begin{gathered}
    B(n_1.i) \supseteq B(n_2.s) \supseteq B(l_1.f) = \emptyset \\
    B(l_0) \supseteq B(n_1.r) = \{ o_0 \}  \\
    B(l_1) \supseteq B(n_2.i) = \{ o_0 \} 
  \end{gathered}
\end{equation}

\begin{equation}
     \Big \{ \{ \hat{r}_5 \} < \{ \hat{r}_0 \} < \{ \hat{r}_1 \} < \{ \hat{r}_6 \} < \{ \hat{r}_4, \hat{r}_7, \hat{r}_2 \} < \{ \hat{r}_3 \}  \Big \}
\end{equation}

\end{frame}

\begin{frame}{Trivial Schedule}{Observation}

Only when \alert{all functions are monotone}. 

$$ \Big\{ \{ r_0, \dots, r_7 \} \Big \}$$

\end{frame}



\begin{frame}[fragile=singleslide]{Example}{CRAG Example with non-monotone function $h$}

\begin{multicols}{3}
\begin{Verbatim}[fontsize=\small]
S -> A B
    local l
    l = h(A.r)
    B.i = h(l)
    A.i = B.s
    S.x = h(l.f)
A -> a
    object o
    o.f <- A.i
    A.r = h(o0)


B -> b
    local l
    l = h(B.i)
    B.s = l.f
\end{Verbatim}
\end{multicols}


\newlinevspace

The trivial schedule \alert{does not work} for this example.

\end{frame}



\begin{frame}{Circular Remote Attribute Grammar}{Schedule Evaluator}

\begin{center}
    \scalebox{0.5}{
    \begin{minipage}{1.3\linewidth}
\begin{algorithm}[H]
\begin{algorithmic}[1]
    \Procedure{\texttt{CRAG\_SCHEDULE\_EVAL}}{$\; (\hat R, \lesssim), \mathit{Val} = \texttt{dict()} \;$}
        \For{$R' \gets (\hat R, \lesssim)$}
            \Do
                \State{$\mathit{change} \gets \texttt{false}$}
                \For{$\hat{r} \gets R'$}
                    \State $\mathit{prev} \gets \texttt{dict()}$
                    \For{$v \gets \hat{r}$}
                        \State{$\mathit{prev}(v) \gets \mathit{Val}(v)$}
                    \EndFor
                    \State \texttt{RAG\_EVAL($\hat{r}, \mathit{Val}$)}
                    \For{$v \gets \hat{r}$}
                        \If{$\mathit{Val}(v) \supset \mathit{prev}(v)$}
                            \State $change \gets \texttt{true}$
                        \EndIf
                    \EndFor
                \EndFor
            \DoWhile{$\mathit{change}$}
        \EndFor
    \EndProcedure
\end{algorithmic}
\caption{Schedule evaluation of circular remote attribute grammar}
\end{algorithm}
    \end{minipage}%
    }    
\end{center}
    
\end{frame}

\begin{frame}{Circular Visit}{Definition}

\begin{definition}\label{def:circular-set-of-occurances}
A visit $V'$, $V' \in \mathscr{V}(p: X \rightarrow \alpha)$, is called \emph{circular} \alert{if the corresponding set in the protocol} of non-terminal in the LHS of production is declared circular. Similarly, a set in protocol is circular if all attributes in that set are declared circular. To indicate a set $\pi$ in protocol is circular, $\pi^c$ is used.
\end{definition}

\end{frame}



\begin{frame}[fragile=singleslide]{Example}{Circular Remote AG}


\begin{figure}
    \centering
\begin{multicols}{3}
\begin{Verbatim}[fontsize=\small]
S -> A B
    local l
    l = A.r
    B.i = l
    A.i = B.s
    S.x = l.f
A -> a
    object o
    o.f <- A.i
    A.r = o


B -> b
    local l
    l = B.i
    B.s = l.f
\end{Verbatim}
\end{multicols}
    \caption{Example of circular remote attribute grammar with all monotone operations.}
    \label{fig:crag-definition-with-all-monotone}
\end{figure}

\end{frame}


\begin{frame}{Example}{Circular visit}

\begin{equation}
\small
\begin{gathered}
\Pi(S) =  \Big \{   \{ S.x \}      \Big \} \\
\Pi(A) =  \Big \{   \{  A.i, A.r \}^c      \Big \} \\
\Pi(B) =  \Big \{   \{  B.i, B.s \}^c      \Big \}
\end{gathered}
\end{equation}

\begin{equation}
\small
\begin{gathered}
\mathscr{V}(p: S \rightarrow A B) = \Big \{  \{  \mathit{visit}(A, 1) <  (l \texttt{=} A.r) < \mathit{visit}(B, 1)  \}    \Big \} \\
\mathscr{V}(p: A \rightarrow a) = \Big \{  \{  (o.f \sqsupseteq A.i)  < (A.r \texttt{=} o)  \}^c    \Big \} \\
\mathscr{V}(p: B \rightarrow b) = \Big \{  \{  (l \texttt{=} B.i) < (B.s \texttt{=} l.f)  \}^c    \Big \}
\end{gathered}
\end{equation}


\end{frame}

\begin{frame}{Fixed-Point Loops}{Case-by-case analysis of when fixed-point loop is needed to avoid nesting fixed-point loops}

{ \small
\begin{tabular}{|p{0.4\linewidth} | p{0.5\linewidth}|}
\hline
% https://tex.stackexchange.com/questions/33486
\multicolumn{1}{|c|}{Visit Type} & \multicolumn{1}{c|}{Fixed-Point Loop?}   \\ 
\hline\hline
Circular visit inside of circular visit & No fixed-point loop. Since the parent visit repeats the evaluation, its loop includes the child visit as well.\\ \hline
Non-circular visit in circular visit & No fixed-point loop. Visit has to evaluate only once. \\ \hline
Non-circular visit in non-circular visit & No fixed-point loop. Visit has to evaluate only once. \\ \hline
Circular visit in non-circular visit & Needs fixed-point loop. \\ \hline
\end{tabular} }

\end{frame}