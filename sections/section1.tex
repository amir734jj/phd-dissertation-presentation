\section{Introduction}

\begin{frame}{Importance of Attribute Grammars}{Applications mainly in compiler construction}
Primary application:

\begin{itemize}
    \item Type Checking
    \item Control-Flow Analysis
    \item Solving problems over AST
\end{itemize}

\emptyline

Other applications:
\begin{itemize}
    \item Natural Language Processing \cite{10.1007/978-3-642-25324-9_25}
    \item Model-driven engineering \cite{schone2020connecting}
    \item Testing Services \cite{habibisharif}
\end{itemize}
\end{frame}


\begin{frame}{Motivation}{Simplest form of attribute grammar}
Math expression consisting of addition \& multiplication

\[ 1 \times 2 + 3  \]

\newlinevspace

How to represent this as a \alert{CFG} with correct order of operation?
\end{frame}


\begin{frame}[fragile=singleslide]{Motivation}{Binary Expression CFG}

How to represent the \alert{final value} as an attribute grammar?

\begin{multicols}{2}
\begin{Verbatim}[fontsize=\small]
E -> E + T
  val = ?

E -> T
  val = ?




T -> T * F
  val = ?

T -> F
  val = ?

F -> digit
  val = ?
\end{Verbatim}
\end{multicols}
\end{frame}


\begin{frame}[fragile=singleslide]{Motivation}{Pure-S AG}

This is an example of \alert{purely synthesized} attribute grammar

\begin{multicols}{2}
\begin{Verbatim}[fontsize=\small]
E -> E + T
  E0.val = E1.val + T.val

E -> T
  E.val = T.val




T -> T * F
  T0.val = T1.val * F.val

T -> F
  T.val = F.val

F -> digit
  F.val = digit.lexical_val
\end{Verbatim}
\end{multicols}

\end{frame}




\begin{frame}{Motivation}{Derivation tree}
Arrows represent \alert{direction} of flow of attribute values

\begin{center}
\scalebox{0.7}{\begin{forest}
  [
    E, name=11
    [E,name=7 [T, name=6 [T,name=3 [F,name=2 [digit, name=1]]  ][*][F, name=5 [digit, name=4]  ]]]
    [+]
    [T, name=10 [F, name=9  [digit, name=8 ]]]
  ]
  \draw[->,dashed] (1) to[out=north west,in=south west] (2);
  \draw[->,dashed] (2) to[out=north west,in=south west] (3);
  \draw[->,dashed] (3) to[out=north west,in=west] (6);
  \draw[->,dashed] (4) to[out=north east,in=south east] (5);
  \draw[->,dashed] (5) to[out=north east,in=east] (6);
  \draw[->,dashed] (6) to[out=north west,in=south west] (7);
  \draw[->,dashed] (8) to[out=north east,in=south east] (9);
  \draw[->,dashed] (9) to[out=north east,in=south east] (10);
  \draw[->,dashed] (7) to[out=north west,in=west] (11);
  \draw[->,dashed] (10) to[out=north east,in=east] (11);
\end{forest}}    
\end{center}

\end{frame}


\begin{frame}{Motivation}{Derivation: Sequence of steps to derive the final form}

How to represent $1 + 2 * 3$ as a derivation?

\begin{equation}
\begin{split}
E \rightarrow E + T
  \rightarrow T + T
  \rightarrow T * F + T
  \rightarrow F * F + T \\
  \rightarrow \mathit{digit} * F + T \\
  \rightarrow \mathit{digit} * \mathit{digit} + T \\
  \rightarrow \mathit{digit} * \mathit{digit} + F \\
  \rightarrow \mathit{digit} * \mathit{digit} + \mathit{digit}
\end{split}
\end{equation}

\end{frame}



\begin{frame}{Attribute Grammars}{Extensions to original Knuth paper}

\begin{itemize}
    \item \alert{Classical Attribute Grammar} Introduced by Knuth in \cite{Knuth68semanticsof}
    \item \alert{Ordered} Attribute Grammar Introduced by Kastens in \cite{10.1007/BF00288644}
    \item \alert{Circular} Attribute Grammar Introduced by Farrow in \cite{10.1145/13310.13320}
    \item \alert{Remote} Attribute Grammar Introduced separately by Boyland in \cite{Boyland05remoteattribute} and Hedin \cite{DBLP:journals/informaticaSI/Hedin00}
    \item \alert{Circular Remote} Attribute Grammar Introduced by Hedin in \cite{10.1016/j.scico.2005.06.005}
\end{itemize}
	
\end{frame}
