\section{Introduction}

\begin{frame}{Importance of Attribute Grammars}{Applications mainly in compiler construction}
Primary application:

\begin{itemize}
    \item Program Analysis
        \begin{itemize}
            \item  Type Checking
            \item Control-Flow Analysis
        \end{itemize}
    \item Solving problems over AST
        \begin{itemize}
            \item  Nullable, First and Follow
        \end{itemize}
\end{itemize}

\emptyline

Other applications:
\begin{itemize}
    \item Natural Language Processing \cite{10.1007/978-3-642-25324-9_25}
    \item Model-driven engineering \cite{schone2020connecting}
    \item Testing Services \cite{habibisharif}
\end{itemize}
\end{frame}

\note[itemize]{
\item Initially attribute grammars were designed and mainly still to this day used in program analysis
\item Generally speaking, attribute grammars can be used in solving problems defined over a tree
\item Recently, however, it has also been used in natural language processing and testing
}


\begin{frame}{Motivation}{Simplest form of attribute grammar}
Math expression consisting of addition \& multiplication

\[ 1 \times 2 + 3  \]

\newlinevspace

How to represent this as a \alert{CFG} with the correct order of operation?
\end{frame}

\note[itemize]{
\item As a motivation, given this simple arithmetic expression, let's utilize attribute grammars to define the semantics of its final value
\item but first let's construct a context-free grammar for it
}

\begin{frame}[fragile=singleslide]{Motivation}{Binary Expression CFG}

How to represent the \alert{final value} as an attribute grammar?

\begin{multicols}{2}
\begin{Verbatim}[fontsize=\small]
E -> E + T
  val = ?

E -> T
  val = ?




T -> T * F
  val = ?

T -> F
  val = ?

F -> digit
  val = ?
\end{Verbatim}
\end{multicols}
\end{frame}


\note[itemize]{
\item This is a context-free grammar describing that arithmetic expression with the correct order of operation
\item How to represent the final value using attribute grammars?
\item To do that we need to define the value of LHS using the values of RHS
}

\begin{frame}[fragile=singleslide]{Motivation}{Pure-S AG}

This is an example of \alert{purely synthesized} attribute grammar

\begin{multicols}{2}
\begin{Verbatim}[fontsize=\small]
E -> E + T
  E0.val = E1.val + T.val

E -> T
  E.val = T.val




T -> T * F
  T0.val = T1.val * F.val

T -> F
  T.val = F.val

F -> digit
  F.val = digit.lexical_val
\end{Verbatim}
\end{multicols}

\end{frame}

\note[itemize]{
\item This is an attribute grammar describing the value of each expression
\item Notice propagation of attribute values is always bottom-up
\item \texttt{val} is a synthesized attribute
\item non-terminal followed by a dot followed by an attribute name is called an attribute occurrence
}


\begin{frame}{Motivation}{Derivation tree}
Arrow ($\to$) represent \alert{direction} of flow of attribute values:

\alert{bottom-up} propagation only in this example

\begin{center}
\scalebox{0.75}{\begin{forest}
  [
    E, name=11
    [E,name=7 [T, name=6 [T,name=3 [F,name=2 [digit, name=1]]  ][*][F, name=5 [digit, name=4]  ]]]
    [+]
    [T, name=10 [F, name=9  [digit, name=8 ]]]
  ]
  \draw[->,dashed] (1) to[out=north west,in=south west] (2);
  \draw[->,dashed] (2) to[out=north west,in=south west] (3);
  \draw[->,dashed] (3) to[out=north west,in=west] (6);
  \draw[->,dashed] (4) to[out=north east,in=south east] (5);
  \draw[->,dashed] (5) to[out=north east,in=east] (6);
  \draw[->,dashed] (6) to[out=north west,in=south west] (7);
  \draw[->,dashed] (8) to[out=north east,in=south east] (9);
  \draw[->,dashed] (9) to[out=north east,in=south east] (10);
  \draw[->,dashed] (7) to[out=north west,in=west] (11);
  \draw[->,dashed] (10) to[out=north east,in=east] (11);
\end{forest}}    
\end{center}

\end{frame}

\note[itemize]{
\item In a practical setup we have a mix of bottom-up and top-down propagation
}



\begin{frame}{Attribute Grammars}{Extensions to the original Knuth paper}

\begin{itemize}
    \item \alert{Classical Attribute Grammar} Introduced by Knuth in \cite{Knuth68semanticsof}
    \item \alert{Ordered} Attribute Grammar Introduced by Kastens in \cite{10.1007/BF00288644}
    \item \alert{Circular} Attribute Grammar Introduced by Farrow in \cite{10.1145/13310.13320}
    \item \alert{Remote} Attribute Grammar Introduced separately by Boyland in \cite{10.5555/924544, Boyland05remoteattribute} and Hedin \cite{DBLP:journals/informaticaSI/Hedin00}
    \item \alert{Circular Remote} Attribute Grammar Introduced by Hedin in \cite{10.1016/j.scico.2005.06.005}
\end{itemize}
	
\end{frame}

\note[itemize]{
\item This is a list of highly influential papers that contributed to the field of attribute grammar and we will be building upon them
\item Starting from Knuth's 1968 paper titled "Semantics of Context-Free Languages"
\item And most recently Hedin 2007 introduced an extension to Knuth's classical attribute grammar called circular remote attribute grammar 
}


\begin{frame}{Goal}{End-goal of this thesis}
    
\begin{itemize}
    \item \alert{Statically scheduling} circular remote attribute grammars
    \begin{itemize}
        \item Extend \alert{definition of $l$-ordered} to CRAG
        \item Suggest a method to \alert{generate a static schedule} for CRAG
        \item Modifying \alert{APS attribute grammar system} to handle this extension
        \item Testing it against circularly defined problems:
        \begin{itemize}
            \item \alert{First} and \alert{Follow} in APS vs JastAdd
            \item \alert{COOL semantic analyzer} in APS using demand vs static evaluation
        \end{itemize}
    \end{itemize}
\end{itemize}
    
\end{frame}

\note[itemize]{
\item What is the end goal of this thesis and how do we measure whether our implementation actually works or not?
\item So, the goal is statically scheduling circular remote attribute grammars
\item To validate that our implementation is actually working, we will modify the APS attribute grammar system and test it against common circularly defined problems over a tree
\item Examples of these circularly defined problems are COOL programming language semantic analyzer, First and Follow in LR parsing
}

\begin{frame}{Impact}{Advancing knowledge}
    
\begin{itemize}
    \item Statically scheduling circular remote AGs has NOT been done before
    \begin{itemize}
        \item Not even static evaluator for \alert{circular AG}
    \end{itemize}
    \item \alert{Complements the attribute grammars} and enables its practical use because of how \alert{common} and \alert{accommodating} circularly defined problems over a tree are in the real-world applications
\end{itemize}    

\end{frame}

\note[itemize]{
\item How this research is going to impact the field of attribute grammar and program semantic analysis?
\item As far as we are aware, statically scheduling circular remote attribute grammars has NOT been done before
\item Not even static evaluator for circular attribute grammars
\item And this thesis brings over static scheduling to the circular remote attribute grammars which is the most accommodating extension of attribute grammars for the users
}