\subsection*{SCC-Chunk Scheduling}{}

\begin{frame}{Methods Overview}{Methods and algorithms}
     \begin{description}
        \item $\checkmark$ Fiber cycle breaking
        \item $\checkmark$  Visit sequences
        \item $\square$  \alert{SCC chunk scheduling}
    \end{description}
\end{frame}

\note[itemize]{
\item Next, we are going to discuss SCC chunk scheduling and how it's working
}


\begin{frame}{Group Scheduling}{Simpler way to static schedule}

In each visit of the visit sequence:
\begin{itemize}
    \item whole group of parent inherited
    \item whole group of parent synthesized
    \item locals (i.e. local attributes and conditionals)
    \item child inherited and synthesized
\end{itemize}
\end{frame}

\note[itemize]{
\item Let's talk about group scheduling, instead of greedy scheduling individual attributes, we can schedule them as a group
\item In each visit of visit sequence there are 4 possible groups: parent inherited, parent synthesized, locals, and child visits (i.e. package of child inherited and synthesized attributes)
\item Attribute is ready to be scheduled when its dependencies are scheduled
\item So instead of checking if an attribute is ready, we say a group of attributes is ready to be scheduled when they are all ready 
\item This greatly simplifies the static schedule generation
}


\begin{frame}{SCC Chunk Scheduling Goal}{What is the end goal?}
    \begin{itemize}
        \item We want to follow group scheduling
        \item We want to preserve scoping for locals (e.g. IF statement)
        \item Circular attributes that are in a cycle with each other need to be scheduled together
    \end{itemize}
\end{frame}

\note[itemize]{
\item An ideal static scheduling algorithm:
\begin{itemize}
    \item Should follow group scheduling
    \item Should preserve locals for scoping purposes
    \item And circular attributes that are involved in a cycle should be scheduled together
\end{itemize}
\item And that is the goal of the SCC chunk scheduling algorithm that we designed
}


\begin{frame}{Chunks}{Chunks as a way of scheduling groups}
    What are the chunks?
    \newlinevspace{}

    \begin{itemize}
        \item \alert{Half-left}: parent inherited attributes for a particular phase
        \item \alert{Half-right}: parent synthesized attributes for a particular phase
        \item \alert{Child visit}: contain child inherited and synthesized for a given phase of a particular child
        \item \alert{Local}: contain a single local attribute
    \end{itemize}
\end{frame}

\note[itemize]{
\item We introduce the concept of chunks
\item There are four types of chunks. Half-left, half-right, visit chunks and locals
}


\begin{frame}{Chunk Graph}{Reflect attribute dependencies into chunk graph}

\begin{figure}
    \centering
\begin{tikzpicture}[%
node distance = 10mm and 12mm,
  start chain = A going below,
   arr/.style = {-Stealth, shorten >=1mm, shorten <=1mm},
   box/.style = {draw, semithick, align=center,
                 minimum height=6mm, text width=23em},
     C/.style = {circle, draw=black!50, semithick, node contents={}},
every join/.style = {arr}
                        ]
    \begin{scope}[nodes={on chain=A}]
\coordinate (aux);
\node[box]  {Create Chunks};   % A-2
\node[box, join]  {Bring Attribute Dependencies to Chunks};
\node[box, join]  {Add Guiding Dependencies to Chunks}; % A-4
\node[box, join]  {Find Strongly Connected Regions in Chunk Graph}; % A-4
    \end{scope}
\end{tikzpicture}
\end{figure}


\end{frame}

\note[itemize]{
\item The first step in our static scheduling algorithm is to create chunks
\item If LHS non-terminal has $n$ phases or inner-sets in its protocol, then we create $n$ half-left and $n$ half-right corresponding to each phase
\item For each non-terminal appearing on the RHS, we construct visit chunks, where each visit chunk has a corresponding child index and phase number
\item Each local attribute belongs to its own chunk
\item Then we reflect attribute dependencies into the chunk graph, basically, if two attributes have a dependency then their belonging chunks have to include that dependency
\item Next we add guiding dependencies because remember some chunks may be empty because we may have added some empty phases in their corresponding protocol
\item These guiding dependencies ensure that visits appear consecutively even though some may be empty
\item Next we find strongly connected components of the chunk graph
}

\begin{frame}{Scheduling Circular vs. Non-Circular Phase}
    \begin{itemize}
        \item \alert{Circular phase}
            \begin{itemize}
                \item One large SCC component containing half-left, half-right, visits and locals and then group scheduling attributes in this SCC
            \end{itemize}
        \item \alert{Non-circular phase}
        \begin{itemize}
            \item Greedy group scheduling attributes 
        \end{itemize}
    \end{itemize}
\end{frame}

\note[itemize]{
 \item We have to distinguish between circular and non-circular parent phases
 \item If the parent phase or protocol inner-set at a particular index is circular, then we should have a single strongly connected component containing both half-left and half-right for this phase, and in the middle of them there should be child visits and locals
 \item If the parent phase is non-circular then we should just greedy schedule attributes
 }

\begin{frame}[fragile=singleslide]{SCC Chunk Scheduling high-level}{Algorithm that makes scheduling CRAG possible}

    \begin{tikzpicture}[%
node distance = 8mm and 12mm,
  start chain = A going below,
   arr/.style = {-Stealth, shorten >=1mm, shorten <=1mm},
   box/.style = {draw, semithick, align=center,
                 minimum height=6mm, text width=9em},
     C/.style = {circle, draw=black!50, semithick, node contents={}},
every join/.style = {arr}
                        ]
    \begin{scope}[nodes={on chain=A}]
\coordinate (aux);
\node[box]  {SCC Scheduling};   % A-2
\node[box, join]  {Chunk Scheduling};
\node[box, join]  {Group Scheduling}; % A-4
    \end{scope}
\node (in)  [C, left =of A-2, 
             label=left: Start];
\node (out) [C, right=of A-4,
             label=right:End];
%%%%
\draw[arr]  (in)  -- (A-2);
\draw[arr]  (A-4) -- (out);

\draw[arr]  (A-3.east) -|-[distance=-6em] (aux) 
                node[pos=0.5, right, align=left] {Loop SCC\\ for each Chunk}
                        -| ($(in)!0.5!(A-2.west)$); 
\draw[arr]  (A-4.west) -|-[distance=-6em] ($(A-2.south)!0.5!(A-3.north)$)
                node[pos=0.5, left, align=right] {Loop Chunk\\ for each Group};

\end{tikzpicture}
\end{frame}

\note[itemize]{
 \item This is the high-level structure of the algorithm
 \item Notice that
 \begin{itemize}
     \item we first find the next ready-to-schedule chunk-component
     \item then we find the next ready-to-schedule chunk
     \item then we apply group scheduling
     \item and then start all over, so it's a 3-step process
 \end{itemize}
 
 \item The full algorithm is written in the methods chapter
 }



\begin{frame}{Fixed-Point Loops}{Case-by-case analysis of when the fixed-point loop is needed or not allowed}

{ \small
\begin{tabular}{|p{0.46\linewidth} | p{0.45\linewidth}|}
\hline
% https://tex.stackexchange.com/questions/33486
\multicolumn{1}{|c|}{Visit Type} & \multicolumn{1}{c|}{Fixed-Point Loop?}   \\ 
\hline\hline
Circular visit inside of circular visit & No fixed-point loop. Since the parent visit repeats the evaluation, its loop includes the child visit as well.\\ \hline
Non-circular visit in a circular visit & Not allowed. \\ \hline
Non-circular visit in a non-circular visit & No fixed-point loop. The visit has to be evaluated only once. \\ \hline
A circular visit in a non-circular visit & Needs a fixed-point loop. \\ \hline
\end{tabular} }

\end{frame}

\note[itemize]{
\item This table describes the case-by-case analysis of when the fixed-point loop is needed during the visit sequence evaluation
\item This is an interesting slide because we are analyzing all possible cases
\item A circular visit in a non-circular visit requires a fixed-point loop
\item Also notice that if a visit is circular and it is called from another circular visit then the fixed-point loop in the outer visit includes the child visit as well. Therefore, no additional fixed-point loop in the child visit is required
\item Additionally, a non-circular visit is NOT allowed in a circular visit
}