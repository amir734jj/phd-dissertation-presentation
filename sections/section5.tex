\section{Implementation}


\begin{frame}{Static Scheduler}{Simplify scheduling by using groups}

Previously:
\begin{itemize}
    \item Old greedy scheduler was \alert{tightly coupled with code generation} module
\end{itemize}
    
Now:
\begin{itemize}
    \item Scheduling is still greedy but it is using groups and is simpler, easier to debug, and \alert{de-coupled from code generation} module
    \item Includes $(\mathit{ph}, \mathit{ch})$ marker to help code generation identify where a \alert{child visit} is happening and mark the \alert{end of parent visit}
\end{itemize}

\end{frame}


\note[itemize]{
\item Lets talk about the implementation of SCC chunk static scheduling in APS
\item The static scheduler or the module that constructs the visit sequences was tightly coupled with the code generation module and it was not working well
\item So we designed a replacement that is still greedy but simpler as it schedules instances as groups and uses visit markers to indicate where child visit happens and when parent visit ends
}




\begin{frame}{Static Schedule Assertions}{Ensuring greedy algorithm generates valid schedule}
    Due to the \alert{greedy nature of the static scheduler}, we included a series of \alert{assertions} to \alert{statically} (NOT during runtime) ensure the generated static schedule makes sense.

\newlinevspace
    
    \begin{itemize}
        \item No non-circular attribute in a circular parent visit
        \item Non-circular declared attributes may not participate in cycles
        \item Parent visits have to appear sequentially
        \item Child visits have to be invoked sequentially
        \item No missing parent and child visit
    \end{itemize}
\end{frame}

\note[itemize]{
\item Due to the greedy nature of our static scheduling algorithm we need some validation or assertions to make sure the generated schedule makes sense  
\item These validations are described in the thesis in the methods chapter
\item But some of these validations are described in this slide
}


\begin{frame}{\texttt{C} Code Statistic}{Basic Statistics About the New Scheduler}
    \begin{itemize}
        \item Old Scheduler only: $\approx500$ lines of \texttt{C} code
        \item New Static Scheduler + \alert{Assertions}: $\approx 3000$ lines of \texttt{C} code
    \end{itemize}
\end{frame}


\note[itemize]{
\item Here we have basic statistics about the new static scheduler algorithm implementation
\item Note that the number on the second line also includes various assertion functions as well
}

\begin{frame}[fragile=singleslide]{CRAG Scala Generated Evaluator}{Attribute Instance Setter + Monotonicity check}
\begin{Verbatim}[fontsize=\scriptsize]
// Template code to support static circular evaluation
// by building upon circular evaluation structure
var changed: Boolean = false;
trait StaticCircularEvaluation[V_P, V_T] extends CircularEvaluation[V_P, V_T] {
  override def set(newValue : ValueType): Unit = {
    val prevValue = value;
    super.set(newValue);
    changed |= prevValue != value;
  }

  override def check(newValue : ValueType): Unit = {
    if (value != null) {
      if (!lattice.v_equal(value, newValue)) {
        if (!lattice.v_compare(value, newValue)) {
          throw new Evaluation.CyclicAttributeException("non-monotonic " + name);
        }
      }
    }
  }
  
  // Do not treat multiple assignments of attributes as errors
  checkForLateUpdate = false;
}
\end{Verbatim}
\end{frame}


\note[itemize]{
\item This is an example of an attribute instance setter, notice that we have a monotonicity checker that compares the old value with the new value to make sure the monotonicity requirement is being followed
\item Here we allow multiple attribute value assignments as this is an attribute that was declared circular 
\item We also have a global changed variable that we set if there was a \enquote{change} in attribute value after the assignment
\item In the next slide we will see how this \enquote{changed} global variable is being used
}

% kill scala codes, especially this
\begin{frame}[fragile=singleslide]{CRAG Scala Generated Evaluator}{Fixed-Point Loop around visits}
\begin{Verbatim}[fontsize=\scriptsize]
// Example of how child visits can be wrapped inside of the do-while loop
// Backup previous (global) changed value and run the code block at least once
val prevChanged_2_0 = changed;
do {
  changed = false;
  visit_4_2(v_prods);
} while (changed);
// Restore the changed value to ensure the fixed-point loop does not interfere
// with potential other fixed-point loops somewhere higher in the tree
changed = prevChanged_2_0;
\end{Verbatim}
\end{frame}

\note[itemize]{
\item Notice that \enquote{changed} global variable is being backed up before we get to the fixed-point loop, and restored after the fixed-point loop
\item And then we see a do-while loop while there is a change somewhere in the attribute instances that are being re-evaluated during the visit subroutine call
}

\begin{frame}{Concession}{Concession in the implementation}
    \alert{Concession} in the implementation was \alert{no support for conditional cycles}

    \newlinevspace 
    There are two situations that can arise when using conditions with circular attribute grammars:
    
    \begin{itemize}
        \item[] \cmark \; \alert{Conditions inside the cycle}, which showcases the need to use direct edges for scheduling chunks because of potential scoping issues (\texttt{cycle-series.aps})
        \item[] \xmark \; Conditional cycles where the \alert{condition sits outside controlling the cycles}, this is where our APS implementation is not supported (\texttt{tiny-coag.aps})
    \end{itemize}
\end{frame}

\note[itemize]{
\item There are two possible types of the conditions with cycles, we can have a condition inside the cycle and condition outside of the cycle. The first one is supported and the second on is not
\item This is a small concession in the implementation that we left-out as a future work
\item There is a graph of an example of such attribute grammar in the thesis
\item In APS, we statically detect if a given attribute grammar has this type of a condition and we throw an error
}
