\section{Benchmarks}


\begin{frame}{Benchmark Workstation}{Desktop PC}


\begin{itemize}
    \item \texttt{i5-8500} CPU
    \item \texttt{32GB} of RAM
    \item \texttt{Ubuntu 22.04-LTS}
    \item \texttt{Scala 2.13.10} 
    \item \alert{JVM flags \texttt{-Xmx16g}}
    \item  No optional optimization flags
\end{itemize}
\end{frame}

\note[itemize]{
\item Next, let's talk about benchmarking our implementation
\item We used a Ubuntu workstation with \texttt{i5} CPU and 32 GB of RAM
\item The highlighted bullet point is a \texttt{16GB} JVM heap size to prevent evaluation running out of memory and repeated garbage collector calls to free the memory
\item We did not use any optional optimization flags
}


\begin{frame}[fragile=singleslide]{First and Follow}{}

\alert{First} and \alert{Follow} are functions defined over a grammar and they are used in \alert{parsing}

\begin{itemize}
    \item to distinguish two productions with the same non-terminal $X$ on the LHS, we examine the First($X$) sets for their corresponding RHS.
    \item set of terminal symbols that can follow a non-terminal $X$ in a parse as Follow(X).
\end{itemize}

\end{frame}

\note[itemize]{
\item First and Follow are functions used in parsing and they are recursively defined and we will use them to validate our implementation of the static evaluator
}



\begin{frame}{Follow Example}{What makes Follow example interesting}
    Why did we choose \textbf{Follow} for benchmarking purposes
    \begin{itemize}
        \item It's a \alert{recursive} problem over a tree
        \begin{itemize}
            \item \alert{circularly defined problem}
        \end{itemize}
        \item Follow was \alert{already implemented} in JastAdd
        \item \alert{Well-known} and \alert{straightforward} to implement
    \end{itemize}
\end{frame}

\note[itemize]{
\item We chose Follow for benchmarking because it is a recursively defined tree problem, and given a context-free grammar with recursively defined productions then calculation of Follow set is going to be interesting 
\item Additionally, the JastAdd team implemented Follow already in their git repository, so if we implement Follow in APS and our implementation for the same context-free grammar runs comparatively close to their implementation, this means we did a good job
}

\begin{frame}[fragile=singleslide]{Example}{Example of First and Follow}

\begin{multicols}{2}
\begin{Verbatim}[fontsize=\small]
S -> id
    | V assign E
V -> id
E -> V
    | num


First(S) = { id }
First(E) = { num, id }
First(V) = { id }

Follow(S) = { }
Follow(E) = { }
Follow(V) = { assign }
\end{Verbatim}
\end{multicols}

\end{frame}

\note[itemize]{
\item These are examples of First and Follow
}

\begin{frame}{Definition}{Formal recursive definition of First and Follow}

\[ \texttt{FIRST}(N) = \bigcup_{N \to \alpha} \texttt{FIRST}(\alpha) \]
\[ \texttt{FIRST}(\epsilon)   = \{ \epsilon \} \]
\[ \texttt{FIRST}(x \beta)   =\{x\} \]
\[ \texttt{FIRST}(N \beta)   =   \texttt{FIRST}(N) \cdot \texttt{FIRST}(\beta)  \]
\[ \alert{\texttt{FOLLOW}}(N) = \bigcup_{N' \to \alpha N \beta} \texttt{FIRST}(\beta) \cdot \alert{\texttt{FOLLOW}}(N') \]

\newlinevspace

Credit: Dr. Boyland's lecture notes
\end{frame}

\note[itemize]{
\item This a formal definition of First and Follow
\item Notice the recursive nature of the problem specifically in the last equation where we have Follow on both sides
}


\begin{frame}[fragile=singleslide]{Test Input}{Consistent test input to be used to benchmark APS and JastAdd}
Input:
    \begin{itemize}
        \item \alert{Randomly generated} recursive context-free grammar fed into both APS and JastAdd
        \item CFG has to \alert{include recursive productions} otherwise it would not be a recursive problem
        \item CFG has to \alert{include epsilon productions} otherwise it would not be interesting
    \end{itemize}

    Example:

\begin{verbatim}
    S -> a S b
    S -> a b
    S -> b a
    S -> epsilon
\end{verbatim}

\end{frame}

\note[itemize]{
\item We fed the same context-free grammar into both APS and JastAdd in-order to calculate their Follow set, and we wanted the randomly generated context-free grammar to be interesting
\item So, we made sure it included epsilon in RHS and a lot of recursive productions
}

\begin{frame}{Follow Example Benchmark}{APS static CRAG schedule vs. JastAdd demand scheduler running Follow}

\begin{figure}[htbp]
        \scalebox{0.95}{
    \tiny
\begin{tblr}
{
rows           = {valign=m},
columns        = {co=-1,halign=c},
hspan          = minimal,
cell{1}{1-4}   = {r=2}{},
cell{1}{5,7,9} = {c=2}{},
cell{8}{5}     = {c=2}{},
cell{7}{9}     = {r=2,c=2}{},
hlines,
vlines,
}
Input size in MB. & Nodes   & \rotatebox{90}{non-terminals} & \rotatebox{90}{terminals} & JastAdd                        &             & \makecell{APS without\\monotonicity check} &             & \makecell{APS with\\monotonicity check} &             \\
                &         &               &           & time in sec.                 & memory in MB. & time in sec.                   & memory in MB. & time in sec.                & memory in MB. \\
2               & 17045   & 256           & 16789     & 0.8                            & 78          & 0.6                              & 63          & 33                            & 98          \\
5               & 67244   & 512           & 66732     & 2.8                            & 137         & 2.3                              & 126         & 348                           & 344         \\
11              & 265762  & 1024          & 264738    & 15                             & 479         & 9.3                              & 388         & 3324                          & 774         \\
21              & 1024167 & 2048          & 1022119   & 127                            & 2793         & 72                               & 1288        & 31716                         & 2946        \\
69              & 3596625 & 4096          & 3592529   & 1193                            & 11073        & 246                              & 5052        & \makecell{skipped\\takes too long\\ to run}                             &             \\
289             & 9075268 & 8192          & 9067076   & \makecell{JVM crashed\\after 1670\\seconds} &             & 861                              & 14362       &                               &             \\
\end{tblr}}
    \caption{Benchmarks of running \href{https://github.com/boyland/aps/blob/master/examples/follow.aps}{\texttt{Follow}} example}
    \label{fig:follow-benchmark}
   
\end{figure}

\end{frame}

\note[itemize]{
\item This table is the time and memory usage 
\item The times are in seconds and memory usage is in MB
\item Notice that if we skip the monotonicity check in APS which is an IF statement to ensure that programmer has not lied to us by declaring function monotone but it is actually non-monotone meaning that attribute value is getting smaller each time the fixed-point loop run, then APS is substantially faster and more memory efficient compared to JastAdd
\item APS trusts that if programmer declares function monotone, its indeed monotone
\item Lets graph this table in log scaling so we can see how much APS static schedule evaluator is faster
}



\pgfplotstableread{
nts  Nodes    ApsTime       ApsMemory        JastAddTime        JastAddMemory
8      17045        0.6    63     0.8     78
9      67244       2.3    126     2.8     137
10      265762      9.3    388     15     479
11      1024167      72    1288     127     2793
12      3596625     246    5052    1193     11073
13      9075268     861    14362    nan       nan
}\followdata

\begin{frame}[fragile=singleslide]{Follow Example Benchmark}{Log scaled time duration}
\begin{figure}[htbp]
\centering
 \scalebox{0.8}{
\begin{tikzpicture}[
    slanted blocks/.style={
        draw, fill=white, font=\tiny\ttfamily, rotate=45,
        anchor=south west, inner sep=2pt
    }]
    \begin{axis}[
        axis line style = {thick, gray},
        ymode = log,
        ymin = .2,
        xtick = {8,9,...,13},
        xlabel = {Non-terminal count},
        ylabel = {Time (seconds)},
        legend style = {nodes={right, font=\scriptsize},
            at={(0.05,0.9)}, anchor=west},
        clip mode = individual,
        grid = major,
        label style={font=\tiny},
        tick label style={font=\tiny},
        %%% Added: how to create the ticks, manual page 343 "Tick Options"
        xticklabel={$2^{\pgfmathprintnumber{\tick}}$},
        ]
        \addplot table[x=nts, y=JastAddTime]{\followdata};
        \addplot table[x=nts, y=ApsTime]{\followdata};
        \legend{JastAdd, APS}
        \pgfplotsinvokeforeach{0,1,...,5}{%%% changed to compile
            \node[slanted blocks] at ({#1+8},.2) %% changed to adjust shift
                {\pgfplotstablegetelem{#1}{Nodes}\of{\followdata}\pgfplotsretval};
        }
        \node[slanted blocks] at (7.1,.2) {\# of nodes};
        \node[starburst, fill=yellow, draw=red, thick, font=\tiny\ttfamily,
            inner sep=0pt, starburst point height=6pt]
            at (12.5, 2000) {crash!};
    \end{axis}
\end{tikzpicture}}
\caption{Graph of time duration in log scaling}
\end{figure}
\end{frame}

\note[itemize]{
\item This is a log-scaled graph of evaluation runtime
\item The shorter evaluation time, the better
\item We see that JastAdd crashes
}

\begin{frame}[fragile=singleslide]{Follow Example Benchmark}{Log scaled memory usage}
\begin{figure}[htbp]
\centering
 \scalebox{0.8}{
\begin{tikzpicture}[
    slanted blocks/.style={
        draw, fill=white, font=\tiny\ttfamily, rotate=45,
        anchor=south west, inner sep=2pt
    }]
    \begin{axis}[
        axis line style = {thick, gray},
        ymode = log,
        ymin = .2,
        xtick = {8,9,...,13},
        xlabel = {Non-terminal count},
        ylabel = {Memory (MB)},
        legend style = {nodes={right, font=\scriptsize},
            at={(0.05,0.9)}, anchor=west},
        clip mode = individual,
        grid = major,
        label style={font=\tiny},
        tick label style={font=\tiny},
        %%% Added: how to create the ticks, manual page 343 "Tick Options"
        xticklabel={$2^{\pgfmathprintnumber{\tick}}$},
        ]
        \addplot table[x=nts, y=JastAddMemory]{\followdata};
        \addplot table[x=nts, y=ApsMemory]{\followdata};
        \legend{JastAdd, APS}
        \pgfplotsinvokeforeach{0,1,...,5}{%%% changed to compile
            \node[slanted blocks] at ({#1+8},.2) %% changed to adjust shift
                {\pgfplotstablegetelem{#1}{Nodes}\of{\followdata}\pgfplotsretval};
        }
        \node[slanted blocks] at (7.1,.2) {\# of nodes};
        \node[starburst, fill=yellow, draw=red, thick, font=\tiny\ttfamily,
            inner sep=0pt, starburst point height=6pt]
            at (12.5, 30000) {crash!};
    \end{axis}
\end{tikzpicture}}

\caption{Graph of memory usage in log scaling}
\end{figure}
\end{frame}

\note[itemize]{
\item This is a log scaled graph of evaluation memory usage
\item The smaller evaluation memory usage, the better
\item We see that JastAdd crashes
}



\begin{frame}{Observations}{Observations after running Follow}
What we can see when comparing APS vs. JastAdd running Follow on the same CFG?

    \begin{itemize}
        \item The \alert{exponential nature of the demand evaluation} in JastAdd becomes evident as the number of the AST nodes gets larger
        \item \alert{Skipping repeated monotonicity validation checks} in APS which happens when it sets the value of a circular attribute instance helps the time complexity
        \item We can see that \alert{APS without a monotonicity check is much faster} than JastAdd but \alert{APS with a monotonicity check is much slower} than JastAdd
    \end{itemize}
\end{frame}

\note[itemize]{
\item The exponential nature of JastAdd's demand evaluation becomes clear as the input gets larger because it has to keep track of runtime dependencies whereas in APS we see a somewhat linear (log-scaled) graph
\item This result was expected. Static scheduling by definition is faster and more efficient than demand evaluation and this benchmark is proof that static scheduling of circular remote attribute grammars is not only possible but it can be done efficiently
}